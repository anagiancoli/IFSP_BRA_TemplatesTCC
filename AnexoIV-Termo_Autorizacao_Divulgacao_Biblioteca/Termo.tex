% Template Termo de Autorização e Divulgação Apêndice B da Biblioteca - Regulamento de TCC criado para IFSP-BRA por profa. Dra. Ana Paula Müller Giancoli - 01/09/2020
% Adequar o arquivo antes de imprimir

\documentclass[12pt,a4paper]{report}
\pagestyle{empty}
\usepackage{helvet} % Usa a fonte Helvet que é parecida com Arial
\renewcommand{\familydefault}{\sfdefault} % define como default.
%\usepackage{lmodern}			% Usa a fonte Latin Modern		
\usepackage[T1]{fontenc}		% Selecao de codigos de fonte.
\usepackage[utf8]{inputenc}		% Codificacao do documento (conversão automática dos acentos)
\usepackage[brazilian]{babel}
\usepackage{gensymb} % para usar simbolo de graus
\usepackage{textcomp} % para usar simbolo de graus\textdregree
\usepackage{graphics}
\usepackage{graphicx,url}
\usepackage[top=4.5cm, bottom=2cm, left=2cm, right=1.5cm,headheight=80pt]{geometry}
\usepackage{fancyhdr}
\pagestyle{fancy}
\usepackage{enumitem} % para colocar (a)
\hyphenpenalty=5000 % para evitar que ultrapasse as margens e faca hifenacao
\tolerance=1000     % para evitar que ultrapasse as margens e faca hifenacao
\usepackage{array}
\newcolumntype{C}[1]{>{\centering\let\newline\\\arraybackslash\hspace{0pt}}m{#1}}

\renewcommand{\headrulewidth}{0pt}
\renewcommand{\footrulewidth}{0.4pt}

\lhead{}
\chead{\includegraphics[scale=0.03]{brasao.jpg} \\ 
\footnotesize MINISTÉRIO DA EDUCAÇÃO \\
INSTITUTO FEDERAL DE EDUCAÇÃO, CIÊNCIA E TECNOLOGIA DE SÃO PAULO
}
\cfoot{	\centering \tiny{Curso de \curso \\IFSP - Campus Bragança Paulista – bra.ifsp.edu.br – Fone: (0xx11) 4034–7800.\\
Av. Major Fernando Valle, 2013 – São Miguel – Bragança Paulista – SP, Brasil – CEP 12903–000} }
\rfoot{\thepage}


\newcommand{\campus}{Câmpus Bragança Paulista}
\newcommand{\cidade}{Bragança Paulista}
\newcommand{\curso}{Tecnologia em Análise e Desenvolvimento de Sistemas}
\newcommand{\cabecalho}{Apêndice B \\ Termo de autorização de Divulgação}
\newcommand{\orientador}{Informar Titulação e nome completo Orientador}
\newcommand{\orientadorA}{Assinatura do Orientador(a)}
% Informar a data por extenso
\newcommand{\local}{Bragança Paulista, 01 de setembro de 2020.}
\newcommand{\tema}{Título do TCC}
\newcommand{\alunoA}{Informar o nome completo aluno I}
\newcommand{\prontuarioA}{Informar prontuário do aluno I}
\newcommand{\alunoAD}{Assinatura do Aluno(a)}

%se mais de um aluno, preecher os demais, caso contrário comente %.
\newcommand{\alunoB}{Informar o nome completo aluno II}
\newcommand{\prontuarioB}{Informar prontuário do aluno II}
%\newcommand{\alunoC}{seu nome completo}
%\newcommand{\prontuarioC}{BP999999}

%ATE AQUI!!! DEPOIS SOMENTE OS ITENS REALIZADOS.

\begin{document}
	\begin{center}
		{\small\textbf{\cabecalho}}\\
		\vspace{0.5cm}
	\end{center}
	
	\noindent Eu, \textbf{\alunoA} e \textbf{\alunoB}, prontuários \textbf{\prontuarioA} e \textbf{\prontuarioB}, respectivamente, alunos do curso de \textbf{\curso}, na qualidade de titular dos direitos morais e patrimoniais da autoria do(a): \vspace{6pt}\\
	( X ) trabalho de conclusão de curso \hspace{0.5cm}   (\hspace{0.5cm}) \hspace{0.5cm} dissertação (\hspace{0.5cm}) tese, que tem por título: \textbf{\tema}, em consonância com as disposições da Lei n\textdegree \hspace{0.01cm} 9.610 de 19 de fevereiro de 1998, autorizo o Instituto Federal de Educação, Ciência e Tecnologia de São Paulo a: \vspace{6pt}\\
	( X ) Incorporar o trabalho ao acervo digital das bibliotecas do IFSP. \vspace{4pt}\\
    (\hspace{0.5cm}) Incorporar o trabalho ao acervo impresso da biblioteca do Câmpus Bragança Paulista-SP.\vspace{4pt} \\
    ( X ) Permitir a consulta, pesquisa e citação do trabalho, desde que citada a fonte.\vspace{4pt} \\
    (\hspace{0.5cm}) Divulgar o trabalho a partir da data:  \hspace{0.01cm}\rule{1cm}{0.1pt} \hspace{0.01cm} / \rule{1cm}{0.1pt}\hspace{0.01cm} / \hspace{0.01cm}\rule{1cm}{0.1pt}.\\
    \textit{\footnotesize{(Obs. O prazo máximo de espera para divulgar o trabalho é de um ano)}}.  \\
    

    \noindent O trabalho está sujeito a registro de patentes e foi encaminhado ao Núcleo de Inovação Tecnológica (NIT) do IFSP?\vspace{6pt} \\
    ( X ) Não.\vspace{6pt} \\
    (\hspace{0.5cm}) Sim. \vspace{6pt}\\



	\begin{table}[ht]
		\centering
		\begin{tabular}{C{10cm}}
		\\
		\cline{1-1}  
		\footnotesize{Prontuário:} \footnotesize{\prontuarioA} - \footnotesize{\alunoA}  
		\\ \scriptsize{\alunoAD}  \\
		\\
		\\
		\\
		\cline{1-1}  
		\footnotesize{Prontuário:} \footnotesize{\prontuarioB} - \footnotesize{\alunoB}  
		\\ \scriptsize{\alunoAD}  \\
		\\
		\\
		\\
		\cline{1-1}  
	    \footnotesize{\orientador} 
		\\ \scriptsize{\orientadorA} 
		\end{tabular}
	\end{table}
	
\vspace{0.5cm}	
\flushright
	\footnotesize{\local} 
	
\end{document}