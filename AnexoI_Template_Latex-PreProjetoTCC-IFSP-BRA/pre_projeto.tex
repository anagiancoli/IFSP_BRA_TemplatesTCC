% Adaptado para IFSP-BRA por Profa. Dra. Ana Paula Müller Giancoli - 27/03/2020

% -- Classe Documento
\documentclass[
	% -- opções da classe memoir --
	hidelinks,
	12pt,				% tamanho da fonte
	openright,			% capítulos começam em pág ímpar (insere página vazia caso preciso)
	oneside,			% para impressão em verso e anverso. Oposto a oneside, twoside
	a4paper,			% tamanho do papel. 
	%normalfigtabnum,
	%pnumromarab,
	% -- opções da classe abntex2 --
	%chapter=TITLE,		% títulos de capítulos convertidos em letras maiúsculas
	%section=TITLE,		% títulos de seções convertidos em letras maiúsculas
	%subsection=TITLE,	% títulos de subseções convertidos em letras maiúsculas
	%subsubsection=TITLE,% títulos de subsubseções convertidos em letras maiúsculas
	% -- opções do pacote babel --
	english,			% idioma adicional para hifenização
	french,				% idioma adicional para hifenização
	spanish,			% idioma adicional para hifenização
	brazil,				% o último idioma é o principal do documento
]{abntex2}

\usepackage{array}
\newcolumntype{P}[1]{>{\centering\arraybackslash}p{#1}}

% --
\usepackage[document]{ragged2e}
% ---
% Pacotes básicos 
% ---
% Usa a fonte Helvet que é parecida com Arial
\usepackage{helvet} 
% define como default.
\renewcommand{\familydefault}{\sfdefault} 
% Selecao de codigos de fonte.
\usepackage[T1]{fontenc}	
% Codificacao do documento (conversão automática dos acentos)
\usepackage[utf8]{inputenc}	
% Usado pela Ficha catalográfica
\usepackage{lastpage}	
% Indenta o primeiro parágrafo de cada seção.
\usepackage{indentfirst}		
\usepackage{graphicx}
\usepackage[table]{xcolor}
\usepackage{float}
\usepackage{lipsum}
\usepackage{tabularx}
\captionstyle{\center} %(para as legendas; use \legend para fonte)
\usepackage{caption}

\usepackage{hyperref}

% ---
% Pacotes de citações
% ---
% Paginas com as citações na bibl
\usepackage[brazilian,hyperpageref]{backref}
% Citações padrão ABNT
\usepackage[alf,abnt-emphasize=bf,abnt-etal-text=emph]{abntex2cite}  
% Adequado para atender a norma ABNT 6023:2018 por Profa. Dra. Ana Paula Müller Giancoli - 27/03/2020
\usepackage{url6023} 

\newcommand{\Csh}{C{\#}}

% -- 
% Pacote para moldura
% --
% Adicionado para criar uma moldura ao redor da página
\usepackage{fancybox} 
% incluído a moldura da página
\fancypage{\setlength{\fboxsep}{12pt}\fbox}{} 
\usepackage{wrapfig}


%--

% comando para inserir autor e ano
\newcommand{\citeauthorandyear}[1]{\citeauthoronline{#1} (\citeyear{#1})}
\renewcommand{\ABNTEXchapterfontsize}{\bfseries\large}
% https://www.tablesgenerator.com/

\usepackage{float}
\floatstyle{plaintop} % Coloca caption no topo
\newfloat{quadro}{htbp}{lop}
\floatname{quadro}{Quadro}
\newcommand{\listofquadros}{\listof{quadro}{Lista de Quadros}}
\renewcommand{\baselinestretch}{1.5}

% --
% Orientador e coorientador
% --

\newcommand{\orient}[1]{\def \oorient {#1}}
\newcommand{\imprimirorient}{\oorient}

\newcommand{\coorient}[1]{\def \ocoorient {#1}}
\newcommand{\imprimircoorient}{\ocoorient}

% ------ ATUALIZAR
\orient{Nome do Orientador com o título}
\coorient{-}


% --
% Autores
% --
\newcommand{\autorA}[1]{\def \oautorA {#1}}
\newcommand{\imprimirautorA}{\oautorA}
\newcommand{\prontuarioA}[1]{\def \oprontuarioA {#1}}
\newcommand{\imprimirprontuarioA}{\oprontuarioA}

\newcommand{\autorB}[1]{\def \oautorB {#1}}
\newcommand{\imprimirautorB}{\oautorB}
\newcommand{\prontuarioB}[1]{\def \oprontuarioB {#1}}
\newcommand{\imprimirprontuarioB}{\oprontuarioB}

\newcommand{\autorC}[1]{\def \oautorC {#1}}
\newcommand{\imprimirautorC}{\oautorC}
\newcommand{\prontuarioC}[1]{\def \oprontuarioC {#1}}
\newcommand{\imprimirprontuarioC}{\oprontuarioC}

% -- Atualizar
% -- Inserir sempre os nomes dos alunos em ordem alfabética
\prontuarioA{BP9999999}
\autorA{Nome do autor A}

% ----- Caso não tenha mais alunos, basta excluir o conteúdo existente dentro das chaves de \prontuarioB{}, \autorB{}, \prontuarioC{}, \autorC{}
\prontuarioB{BP9999999}
\autorB{Nome do autor B}

\prontuarioC{BP9999999}
\autorC{Nome do autor C}

% ----

% --
% Inicio do Documento
% --

\begin{document}
\selectlanguage{brazil}
\thispagestyle{empty} % remove a numeração desta página
\textual

\begin{wrapfigure}[2]{l}{0.30\textwidth}
\includegraphics[scale=.35]{imagens/IFSP-BRA.png} 
\end{wrapfigure}

\centering
\footnotesize\textbf{Tecnologia em Análise e Desenvolvimento de Sistemas} \\	\footnotesize\textbf{Metodologia de Pesquisa Científica e tecnológica - MPCI4}\\

\titulo{Proposta de Trabalho de Conclusão de Curso (TCC)}
\begin{center}
\vspace{1cm}
    \normalsize \textbf{\imprimirtitulo}
\end{center}
			
\begin{flushleft}
    \small \textbf{1. COMPOSIÇÃO DO GRUPO DE TRABALHO}
\end{flushleft}
\begin{center}
	\begin{tabular}{|m{0.5cm}|m{4cm}|m{10cm}|}
		\hline 
	    \textbf{n$^o$.} & \multicolumn{2}{c|}{\textbf{Nome do aluno}} \\
		\hline
		\textbf{1} & \textit{\imprimirprontuarioA} & \textit{\imprimirautorA}\\
		\hline
		\textbf{2} & \textit{\imprimirprontuarioB} & \textit{\imprimirautorB}\\
		\hline
		\textbf{3} & \textit{\imprimirprontuarioC} & \textit{\imprimirautorC}\\
		\hline
	\end{tabular}
\end{center}
	
\begin{flushleft}
   \small \textbf{1.1. PROFESSOR(ES) ORIENTADOR(ES) }
\end{flushleft}

\begin{center}
	\begin{tabular}{|m{5cm}|m{10cm}|}
	    \hline 
	    \multicolumn{2}{|c|}{\textbf{Nome}} \\
		\hline
		\textbf{Orientador} & \textit{\imprimirorient} \\
		\hline
		\textbf{Coorientador} & \textit{\imprimircoorient} \\
		\hline
	\end{tabular}
\end{center}

\begin{flushleft}
   \normalsize\textbf{2.TÍTULO E PROPOSTA DO PROJETO}
\end{flushleft}

\begin{flushleft}
   \small \textbf{Título do Projeto:}
\end{flushleft} 
\normalsize
% --- Alterar o Título do Projeto:
Indicar aqui o Título do projeto\\

\begin{flushleft}
\textbf{Problema de pesquisa (pergunta):}
\end{flushleft} 
% --- Para inserir o problema de pesquisa, localize o arquivo e altere
\justify


\hspace{0.5cm} Contextualizar e efetuar uma pergunta que será respondida por meio do desenvolvimento da aplicação.


\begin{flushleft}
\textbf{Justificativa (relevância do estudo):}
\end{flushleft} 
% --- Para inserir a justificativa, localize o arquivo e altere
\justify

\hspace{0.5cm} Por meio de citação de autores, contextualizar e justificar o porquê que o seu projeto é importante.

\begin{flushleft}
\textbf{Objetivo Geral:}
\end{flushleft} 
% --- Para inserir o objetivo geral, localize o arquivo e altere
\justify

O objetivo geral do projeto é desenvolver ....
Refere-se de uma maneira global o que o aplicativo ou plataforma ou aplicação terá.


\begin{flushleft}
\textbf{Objetivos Específicos:}
\end{flushleft} 
% --- Para inserir os objetivos especificos, localize o arquivo e altere
\justify

\begin{itemize}
	\item Objetivo 1;
	\item Objetivo 2;
	\item Objetivo n.
\end{itemize}

\begin{justify}
\textbf{Método de Pesquisa (definição da linha metodológica coerente com o problema:}
\end{justify} 
% --- Para inserir o metodo, localize o arquivo e altere
\justify

\hspace{0.5cm} É necessário ...

\begin{justify}
\textbf{Metodologia do Desenvolvimento do Sistema (descrição do processo de desenvolvimento, objeto de estudo, tecnologias, requisitos, ferramentas e procedimentos de elaboração e de análise do sistema):}
\end{justify} 
% --- Para inserir a metodologia, localize o arquivo e altere
\subsection{Metodologia do Desenvolvimento do Sistema}
 Descrição do processo de desenvolvimento, objeto de estudo, tecnologias, requisitos, ferramentas e procedimentos de elaboração e de análise de sistemas.

Desenvolver ...




\begin{flushleft}
\textbf{Resultados esperados:}
\end{flushleft} 
% --- Para inserir os resultados, localize o arquivo e altere
\justify
\hspace{0.5cm} Descrever o que se espera com a criação do projeto.

\begin{flushleft}
\textbf{Cronograma (considerar o 5$^o$. e 6$^o$. módulos:}
\end{flushleft} 
% --- Para inserir o cronograma, localize o arquivo e altere
\justify
Considerar os períodos 4, 5 e 6.
O Quadro~\ref{qua:cronograma} resume o cronograma de atividades da proposta de trabalho de conclusão de curso (TCC).


\begin{quadro}[H]
    \begin{center}
	    \caption{Cronograma de execução de atividades por semestre.}
		\label{qua:cronograma}
		\begin{tabular}{|m{3.5cm}|m{.2cm}|m{.2cm}|m{.2cm}|m{.2cm}|m{.2cm}|m{.2cm}|m{.2cm}|m{.2cm}|m{.2cm}|m{.2cm}|m{.2cm}|m{.2cm}|m{.2cm}|m{.2cm}|m{.2cm}|m{.2cm}|m{.2cm}|m{.2cm}|}
			\hline
			\centering
			\cellcolor{gray!50}\textbf{Entregáveis} & \multicolumn{18}{c|}{ \cellcolor{gray!50}\textbf{Atividades a serem realizadas / mês}}\\
			\hline
		    \centering
			\textbf{Anos} & \multicolumn{12}{c|}{\textbf{2022}} & \multicolumn{6}{c|}{\textbf{2023}} \\
			\hline
			\centering
            \cellcolor{gray!50}\textbf{Meses}&\cellcolor{gray!50}\textbf{1}&\cellcolor{gray!50}\textbf{2}&\cellcolor{gray!50}\textbf{3}&\cellcolor{gray!50}\textbf{4}&\cellcolor{gray!50}\textbf{5}&\cellcolor{gray!50}\textbf{6}&\cellcolor{gray!50}\textbf{7}&\cellcolor{gray!50}\textbf{8}&\cellcolor{gray!50}\textbf{9}&\cellcolor{gray!50}\textbf{10}&\cellcolor{gray!50}\textbf{11}&\cellcolor{gray!50}\textbf{12}&\cellcolor{gray!50}\textbf{1}&\cellcolor{gray!50}\textbf{2}&\cellcolor{gray!50}\textbf{3}&\cellcolor{gray!50}\textbf{4} &\cellcolor{gray!50}\textbf{5} &\cellcolor{gray!50}\textbf{6}\\
			\hline
			Formação do Grupo &\cellcolor{gray!50} & & & & & & & & & & & & & & & & & \\
			\hline
			Definição do tema &\cellcolor{gray!50} & & & & & & & & & & & & & & & & & \\
			\hline
			Introdução e problema &\cellcolor{gray!50} & & & & & & & & & & & & & & & & & \\
			\hline
			Justificativa e Objetivos & &\cellcolor{gray!50} & & & & & & & & & & & & & & & & \\
			\hline
			Pesquisa de metodologias & & &\cellcolor{gray!50} &\cellcolor{gray!50} & & & & & & & & & & & & & & \\
			\hline
			Resultados esperados & & & & &\cellcolor{gray!50} & & & & & & & & & & & & &\\
			\hline
			Cronograma e Apresentação & & & & &\cellcolor{gray!50} & & & & & & & & & & & & & \\
			\hline
			Revisão Bibliográfica & & & & & &  \cellcolor{gray!50} &\cellcolor{gray!50} &\cellcolor{gray!50} & \cellcolor{gray!50} & & & & & & & & & \\
			\hline
			Levantamento de Requisitos  & & & & & & &\cellcolor{gray!50} &\cellcolor{gray!50} &\cellcolor{gray!50}& & & & & & & & & \\
			\hline
			Elaborar Diagrama de Casos de uso e Classes  & & & & & & & & &\cellcolor{gray!50} &\cellcolor{gray!50} & & & & & & & & \\
			\hline
			Descrições dos Casos de Uso  & & & & & & & & &\cellcolor{gray!50} &\cellcolor{gray!50} & & & & & & & & \\
			\hline
			Elaboração dos Protótipos em ferramenta específica & & & & & & & & & &\cellcolor{gray!50} &\cellcolor{gray!50} &  \cellcolor{gray!50} & & & & & & \\
			\hline
			Entregas e Revisão & & & & & & & & & & & & & &\cellcolor{gray!50} &\cellcolor{gray!50} &\cellcolor{gray!50} & & \\ 
			\hline
			Desenvolvimento  & & & & & & &\cellcolor{gray!50} &\cellcolor{gray!50} &\cellcolor{gray!50} &\cellcolor{gray!50} &\cellcolor{gray!50} &\cellcolor{gray!50} &\cellcolor{gray!50} &\cellcolor{gray!50} &\cellcolor{gray!50} &\cellcolor{gray!50} &\cellcolor{gray!50} & \\
			\hline
			Finalização da documentação  & & & & & & & & & & & & & & & & &\cellcolor{gray!50} & \\
			\hline
			Defesa do TCC & & & & & & & & & & & & & & & & & &\cellcolor{gray!50} \\
			\hline
		\end{tabular}
		\vspace{0.5cm}	\\Fonte: Autoria própria
	\end{center}
\end{quadro}


\bibliography{referencias}
\vspace{1cm}

\begin{center}
	\begin{tabular}{|m{10cm}|m{5cm}|}
		\hline
		\textbf{Alunos} & \textbf{Assinatura} \\
		\hline
		 \textit{\imprimirprontuarioA - \imprimirautorA} &  \\
		\hline
		\textit{\imprimirprontuarioB - \imprimirautorB} &  \\
		\hline
		\textit{\imprimirprontuarioC - \imprimirautorC} & \\
		\hline
	\end{tabular}
\end{center}

\vspace{0.5cm}

\begin{center}
	\begin{tabular}{|m{10cm}|m{5cm}|}
		\hline
		\textbf{Orientador(es)} & \textbf{Assinatura} \\
		\hline
		\textit{\imprimirorient} & \\
		\hline
		\textit{\imprimircoorient} & \\
		\hline
	\end{tabular}
\end{center}

% ---
% Para gerar a versão FINAL, comentar essa chamada %

Inserir um percentual na frente para comentar e não aparecer na compilação do documento final. 

No arquivo pre\_projeto.tex --- "\ input{exemplos}"

Texto considerando a revisão da literatura pertinente, dividido em seções e subseções.

Este é um exemplo de como usar figuras. Referência cruzada: Figura~\ref{fig:exemplo}


\begin{figure}[!htbp]
	\centering
	\caption{Exemplo de figura}
	%scale redimensiona a figura.
	%1.5 = 150% do tamanho original
	%1 = 100% do tamanho original
	%0.20 = 20% do tamanho original
	\includegraphics[scale=1]{imagens/IFSP-BRA.png}
	\legend{Fonte: Disponível em: http://bra.ifsp.edu.br. Acesso em 27 mar. 2020}.
	\label{fig:exemplo}
\end{figure}



Este é um exemplo de como usar tabelas. Referência cruzada: Tabela~\ref{tab:exemplo}


\begin{table}[!htbp]
\centering
\caption{Exemplo de tabela de 2 colunas}
	\begin{tabular}{ c | c }
		\hline
		\textbf{Coluna 1} & \textbf{Coluna 2} \\ \hline
		Dado 1a           & Dado 2a           \\ \hline
		Dado 1b           & Dado 2b           \\ \hline
		Dado 1c           & Dado 2c           \\ \hline
		Dado 1d           & Dado 2d           \\ \hline
	\end{tabular}
	\\ \vspace{0.2cm}
	Fonte: Autoria própria
	\label{tab:exemplo}
\end{table}



Este é um exemplo de como usar quadros. Referência cruzada: Quadro~\ref{tab:exemploquad}


\begin{quadro}[!htbp]
\centering
\caption{Exemplo de Quadro de 3 colunas}
	\begin{tabular}{ | m{10em} | m{4cm}| m{4cm} | }
		\hline
		\textbf{Coluna 1} & \textbf{Coluna 2} & \textbf{Coluna 3} \\ \hline
		Dado 1a           & Dado 2a & \\ \hline
		Dado 1b           & Dado 2b & \\ \hline
		Dado 1c           & Dado 2c & \\ \hline
		Dado 1d           & Dado 2d & \\ \hline
	\end{tabular}
	\\ \vspace{0.2cm}
	Fonte: Autoria própria
	\label{tab:exemploquad}
\end{quadro}


Este é um exemplo de como usar quadros. Referência cruzada: Quadro~\ref{tab:exemplo2}


\begin{quadro}[!htbp]
\centering
\caption{Exemplo de Quadro de 2 colunas}
	\begin{tabular}{ | m{10em} | m{4cm}| }
		\hline
		\textbf{Coluna 1} & \textbf{Coluna 2}  \\ \hline
		Dado 1a           & Dado 2a  \\ \hline
		Dado 1b           & Dado 2b  \\ \hline
		Dado 1c           & Dado 2c  \\ \hline
		Dado 1d           & Dado 2d  \\ \hline
	\end{tabular}
	\\ \vspace{0.2cm}
	Fonte: Autoria própria
	\label{tab:exemplo2}
\end{quadro}


Este é um exemplo de como usar equações. Referência cruzada: Equação~\ref{eq:exemplo}

\begin{equation}
\sum_{i=1}^{n} i = \frac{n(n+1)}{2}
\label{eq:exemplo}
\end{equation}

\clearpage


Este é um exemplo de como inserir texto sem formatação (ambiente verbatim):

\begin{verbatim}
	Texto sem formatação, como espaçamento igual.
\end{verbatim}


Exemplo de lista de itens:

\begin{itemize}
	\item \textbf{Item 1:} texto...;
	\item \textbf{Item 2:} texto...;
    \begin{itemize}
            \item \textbf{Subitem:} texto...;
            \item \textbf{Subitem:} texto...;
            \item \textbf{Subitem:} texto...;
        \end{itemize}
	\item \textbf{Item 3:} texto...;
	\item \textbf{Item n:} texto....
\end{itemize}


Exemplo de lista numerada:

\begin{enumerate}
	\item \textbf{Item:} texto...;
	\item \textbf{Item:} texto...;
    \begin{enumerate}
        \item \textbf{Subitem:} texto...;
        \item \textbf{Subitem:} texto...;
        \item \textbf{Subitem:} texto...;
    \end{enumerate}
	\item \textbf{Item:} texto...;
	\item \textbf{Item:} texto....
\end{enumerate}

Tipos de referência a serem utilizados no arquivo referencias.bib:
Exemplos de estruturas.\\

@article{<citation key>,
    author        = {},
    title         = {},
    journaltitle  = {},
    year          = {}
}

@online{<citation key>,
    author        = {},
    title         = {},
    year          = {},
    url           = {},
    urlaccessdate = {}
}

@book{<citation key>,
    author        = {},
    title         = {},
    year          = {}
}

@misc{<citation key>,
    author        = {},
    title         = {},
    year          = {},
    url           = {},
    urlaccessdate = {}
}

Exemplos de Notas de rodapé de leis: \\

... instituído pela Lei Complementar Nº 128\footnote{BRASIL. LEI COMPLEMENTAR Nº 128, de 19 de dezembro de 2008. Altera a Lei Complementar Nº 123, de 14 de dezembro de 2006, altera as Leis nos 8.212, de 24 de julho de 1991, 8.213, de 24 de julho de 1991, 10.406, de 10 de janeiro de 2002 – Código Civil, 8.029, de 12 de abril de 1990, e dá outras providências. \textbf{Diário Oficial [da] República Federativa do Brasil}, Brasília, DF, 19 dez. 2008. Disponível em: http://www.planalto.gov.br/ccivil\_03/leis/lcp/lcp128.htm. Acesso em: 24 nov 2021.}
\\
\\

Exemplos de comandos para texto e referências:

\begin{itemize}
	\item Para iniciar um novo parágrafo, basta deixar uma linha em branco no código fonte;
	\item Não force o compilador a pular mais de uma linha, pois terá influência negativa na composição do documento;
	\item Sempre deixe o \LaTeX\ realizar a formatação de parágrafos e posicionamento de elementos;
	\item Utilização de aspas simples (abertura \verb|`|, fechamento \verb|'|): `Texto entre aspas simples';
	\item Utilização de aspas duplas (abertura \verb|``|, fechamento \verb|''|): ``Texto entre aspas duplas'';
	\item Negrito (comando \verb|\textbf|): \textbf{texto em negrito};
	\item Itálico (comando \verb|\textit|): \textit{texto em itálico};
	\item Sublinhado (comando \verb|\underline|): \underline{texto sublinhado};
	\item Negrito e itálico (usar comandos juntos): \textbf{\textit{texto em negrito e itálico}};
	\item Alterar cor do texto (comando \verb|\textcolor{cor}{texto}|):
	\begin{itemize}
		\item Exemplo \verb|\textcolor{red}{texto}|: \textcolor{red}{texto vermelho};
		\item Exemplo \verb|\textcolor[RGB]{255, 102, 0}|: \textcolor[RGB]{255, 102, 0}{texto laranja};
		\item Exemplo \verb|\textcolor[HTML]{006AD7}|: \textcolor[HTML]{006AD7}{texto azul};
	\end{itemize}
	\item Ambiente matemático inline (comando \verb|$ expressão $|): $s = x^2-2x +1$;
	\item Referência normal (comando \verb|\cite|):
	\begin{itemize}
		\item \cite{Agaisse1995};
		\item \cite{Abedi2014};
		\item \cite{Baum2016};
	\end{itemize}
	\item Referência normal com mais de uma obra (comando \verb|\cite|):
	\begin{itemize}
		\item \cite{Abedi2014, Agaisse1995};
		\item \cite{AgapitoTenfen2014, Baum2016, Nelson2014};
	\end{itemize}
	\item Referência nome e ano (comando \verb|\citeauthorandyear|):
	\begin{itemize}
		\item \citeauthorandyear{bervian2007a};
		\item \citeonline{bervian2007a};
		\item \citeauthorandyear{documento2018};
		\item \citeauthorandyear{Abedi2014};
	\end{itemize}
	\item Referência apud ({mais antigo} {mais novo}):\\
	\verb|\apud[pagina]{autor-indireto}{autor-direto}|):
	\begin{itemize}
		\item \apud{Agaisse1995}{Abedi2014};
		\item \apudonline{Agaisse1995}{Abedi2014};
	\end{itemize}
	\item Referência apud [pagina]({mais antigo} {mais novo}):\\
	\verb|\apud[pagina]{autor-indireto}{autor-direto}|):
	\begin{itemize}
		\item \apud[p. 81]{Agaisse1995}{Abedi2014};
		\item \apudonline[p. 81]{Agaisse1995}{Abedi2014};
	\end{itemize}
	
	\item Referência do mesmo autor, mesmo ano com obras distintas 
	\verb|\citeauthorandyear|):
	\begin{itemize}
		\item \citeauthorandyear{Agaisse1996a};
		\item \citeauthorandyear{Agaisse1996b};
		\item \cite{Agaisse1996a};
		\item \cite{Agaisse1996b};
	\end{itemize}
	
	
\end{itemize}


Exemplo 1 de citação direta:

\begin{citacao}
	Os 20 aminoácidos usualmente encontrados como resíduos em proteínas contém um grupo $\alpha$-carboxil, um grupo $\alpha$-amino e um grupo R distinto substituído no átomo de carbono $\alpha$. O átomo de carbono $\alpha$ de todos os aminoácidos, com exceção da glicina, é assimétrico e, portanto, os aminoácidos podem existir em pelo menos duas formas estereoisoméricas. Somente os estereoisômeros L, com uma configuração relacionada à configuração absoluta da molécula de referência L-gliceraldeído, são encontrados em proteínas \cite[p. 81]{Nelson2014}.
\end{citacao}

Exemplo 2 de citação direta:

\begin{citacao}
	\textit{These various insecticidal proteins are synthesized during the stationary phase and accumulate in the mother cell as a crystal inclusion which can account for up to 25\% of the dry weight of the sporulated cells. The amount of crystal protein produced by a B. thuringiensis culture in laboratory conditions (about 0.5 mg of protein per ml) and the size of the crystals (24) indicate that each cell has to synthesize $10^6$ to $2 \times 10^6$ $\delta$-endotoxin molecules during the stationary phase to form a crystal} \cite[p. 1]{Agaisse1995}.
\end{citacao}

Exemplo de nota de rodapé\footnote{Essa é uma nota de rodapé!}.

% Inserir um percentual na frente para comentar e não aparecer na compilação do documento final.


Inserir um percentual na frente para comentar e não aparecer na compilação do documento final. 

No arquivo pre\_projeto.tex --- "\ input{exemplos}"

Texto considerando a revisão da literatura pertinente, dividido em seções e subseções.

Este é um exemplo de como usar figuras. Referência cruzada: Figura~\ref{fig:exemplo}


\begin{figure}[!htbp]
	\centering
	\caption{Exemplo de figura}
	%scale redimensiona a figura.
	%1.5 = 150% do tamanho original
	%1 = 100% do tamanho original
	%0.20 = 20% do tamanho original
	\includegraphics[scale=1]{imagens/IFSP-BRA.png}
	\legend{Fonte: Disponível em: http://bra.ifsp.edu.br. Acesso em 27 mar. 2020}.
	\label{fig:exemplo}
\end{figure}



Este é um exemplo de como usar tabelas. Referência cruzada: Tabela~\ref{tab:exemplo}


\begin{table}[!htbp]
\centering
\caption{Exemplo de tabela de 2 colunas}
	\begin{tabular}{ c | c }
		\hline
		\textbf{Coluna 1} & \textbf{Coluna 2} \\ \hline
		Dado 1a           & Dado 2a           \\ \hline
		Dado 1b           & Dado 2b           \\ \hline
		Dado 1c           & Dado 2c           \\ \hline
		Dado 1d           & Dado 2d           \\ \hline
	\end{tabular}
	\\ \vspace{0.2cm}
	Fonte: Autoria própria
	\label{tab:exemplo}
\end{table}



Este é um exemplo de como usar quadros. Referência cruzada: Quadro~\ref{tab:exemploquad}


\begin{quadro}[!htbp]
\centering
\caption{Exemplo de Quadro de 3 colunas}
	\begin{tabular}{ | m{10em} | m{4cm}| m{4cm} | }
		\hline
		\textbf{Coluna 1} & \textbf{Coluna 2} & \textbf{Coluna 3} \\ \hline
		Dado 1a           & Dado 2a & \\ \hline
		Dado 1b           & Dado 2b & \\ \hline
		Dado 1c           & Dado 2c & \\ \hline
		Dado 1d           & Dado 2d & \\ \hline
	\end{tabular}
	\\ \vspace{0.2cm}
	Fonte: Autoria própria
	\label{tab:exemploquad}
\end{quadro}


Este é um exemplo de como usar quadros. Referência cruzada: Quadro~\ref{tab:exemplo2}


\begin{quadro}[!htbp]
\centering
\caption{Exemplo de Quadro de 2 colunas}
	\begin{tabular}{ | m{10em} | m{4cm}| }
		\hline
		\textbf{Coluna 1} & \textbf{Coluna 2}  \\ \hline
		Dado 1a           & Dado 2a  \\ \hline
		Dado 1b           & Dado 2b  \\ \hline
		Dado 1c           & Dado 2c  \\ \hline
		Dado 1d           & Dado 2d  \\ \hline
	\end{tabular}
	\\ \vspace{0.2cm}
	Fonte: Autoria própria
	\label{tab:exemplo2}
\end{quadro}


Este é um exemplo de como usar equações. Referência cruzada: Equação~\ref{eq:exemplo}

\begin{equation}
\sum_{i=1}^{n} i = \frac{n(n+1)}{2}
\label{eq:exemplo}
\end{equation}

\clearpage


Este é um exemplo de como inserir texto sem formatação (ambiente verbatim):

\begin{verbatim}
	Texto sem formatação, como espaçamento igual.
\end{verbatim}


Exemplo de lista de itens:

\begin{itemize}
	\item \textbf{Item 1:} texto...;
	\item \textbf{Item 2:} texto...;
    \begin{itemize}
            \item \textbf{Subitem:} texto...;
            \item \textbf{Subitem:} texto...;
            \item \textbf{Subitem:} texto...;
        \end{itemize}
	\item \textbf{Item 3:} texto...;
	\item \textbf{Item n:} texto....
\end{itemize}


Exemplo de lista numerada:

\begin{enumerate}
	\item \textbf{Item:} texto...;
	\item \textbf{Item:} texto...;
    \begin{enumerate}
        \item \textbf{Subitem:} texto...;
        \item \textbf{Subitem:} texto...;
        \item \textbf{Subitem:} texto...;
    \end{enumerate}
	\item \textbf{Item:} texto...;
	\item \textbf{Item:} texto....
\end{enumerate}

Tipos de referência a serem utilizados no arquivo referencias.bib:
Exemplos de estruturas.\\

@article{<citation key>,
    author        = {},
    title         = {},
    journaltitle  = {},
    year          = {}
}

@online{<citation key>,
    author        = {},
    title         = {},
    year          = {},
    url           = {},
    urlaccessdate = {}
}

@book{<citation key>,
    author        = {},
    title         = {},
    year          = {}
}

@misc{<citation key>,
    author        = {},
    title         = {},
    year          = {},
    url           = {},
    urlaccessdate = {}
}

Exemplos de Notas de rodapé de leis: \\

... instituído pela Lei Complementar Nº 128\footnote{BRASIL. LEI COMPLEMENTAR Nº 128, de 19 de dezembro de 2008. Altera a Lei Complementar Nº 123, de 14 de dezembro de 2006, altera as Leis nos 8.212, de 24 de julho de 1991, 8.213, de 24 de julho de 1991, 10.406, de 10 de janeiro de 2002 – Código Civil, 8.029, de 12 de abril de 1990, e dá outras providências. \textbf{Diário Oficial [da] República Federativa do Brasil}, Brasília, DF, 19 dez. 2008. Disponível em: http://www.planalto.gov.br/ccivil\_03/leis/lcp/lcp128.htm. Acesso em: 24 nov 2021.}
\\
\\

Exemplos de comandos para texto e referências:

\begin{itemize}
	\item Para iniciar um novo parágrafo, basta deixar uma linha em branco no código fonte;
	\item Não force o compilador a pular mais de uma linha, pois terá influência negativa na composição do documento;
	\item Sempre deixe o \LaTeX\ realizar a formatação de parágrafos e posicionamento de elementos;
	\item Utilização de aspas simples (abertura \verb|`|, fechamento \verb|'|): `Texto entre aspas simples';
	\item Utilização de aspas duplas (abertura \verb|``|, fechamento \verb|''|): ``Texto entre aspas duplas'';
	\item Negrito (comando \verb|\textbf|): \textbf{texto em negrito};
	\item Itálico (comando \verb|\textit|): \textit{texto em itálico};
	\item Sublinhado (comando \verb|\underline|): \underline{texto sublinhado};
	\item Negrito e itálico (usar comandos juntos): \textbf{\textit{texto em negrito e itálico}};
	\item Alterar cor do texto (comando \verb|\textcolor{cor}{texto}|):
	\begin{itemize}
		\item Exemplo \verb|\textcolor{red}{texto}|: \textcolor{red}{texto vermelho};
		\item Exemplo \verb|\textcolor[RGB]{255, 102, 0}|: \textcolor[RGB]{255, 102, 0}{texto laranja};
		\item Exemplo \verb|\textcolor[HTML]{006AD7}|: \textcolor[HTML]{006AD7}{texto azul};
	\end{itemize}
	\item Ambiente matemático inline (comando \verb|$ expressão $|): $s = x^2-2x +1$;
	\item Referência normal (comando \verb|\cite|):
	\begin{itemize}
		\item \cite{Agaisse1995};
		\item \cite{Abedi2014};
		\item \cite{Baum2016};
	\end{itemize}
	\item Referência normal com mais de uma obra (comando \verb|\cite|):
	\begin{itemize}
		\item \cite{Abedi2014, Agaisse1995};
		\item \cite{AgapitoTenfen2014, Baum2016, Nelson2014};
	\end{itemize}
	\item Referência nome e ano (comando \verb|\citeauthorandyear|):
	\begin{itemize}
		\item \citeauthorandyear{bervian2007a};
		\item \citeonline{bervian2007a};
		\item \citeauthorandyear{documento2018};
		\item \citeauthorandyear{Abedi2014};
	\end{itemize}
	\item Referência apud ({mais antigo} {mais novo}):\\
	\verb|\apud[pagina]{autor-indireto}{autor-direto}|):
	\begin{itemize}
		\item \apud{Agaisse1995}{Abedi2014};
		\item \apudonline{Agaisse1995}{Abedi2014};
	\end{itemize}
	\item Referência apud [pagina]({mais antigo} {mais novo}):\\
	\verb|\apud[pagina]{autor-indireto}{autor-direto}|):
	\begin{itemize}
		\item \apud[p. 81]{Agaisse1995}{Abedi2014};
		\item \apudonline[p. 81]{Agaisse1995}{Abedi2014};
	\end{itemize}
	
	\item Referência do mesmo autor, mesmo ano com obras distintas 
	\verb|\citeauthorandyear|):
	\begin{itemize}
		\item \citeauthorandyear{Agaisse1996a};
		\item \citeauthorandyear{Agaisse1996b};
		\item \cite{Agaisse1996a};
		\item \cite{Agaisse1996b};
	\end{itemize}
	
	
\end{itemize}


Exemplo 1 de citação direta:

\begin{citacao}
	Os 20 aminoácidos usualmente encontrados como resíduos em proteínas contém um grupo $\alpha$-carboxil, um grupo $\alpha$-amino e um grupo R distinto substituído no átomo de carbono $\alpha$. O átomo de carbono $\alpha$ de todos os aminoácidos, com exceção da glicina, é assimétrico e, portanto, os aminoácidos podem existir em pelo menos duas formas estereoisoméricas. Somente os estereoisômeros L, com uma configuração relacionada à configuração absoluta da molécula de referência L-gliceraldeído, são encontrados em proteínas \cite[p. 81]{Nelson2014}.
\end{citacao}

Exemplo 2 de citação direta:

\begin{citacao}
	\textit{These various insecticidal proteins are synthesized during the stationary phase and accumulate in the mother cell as a crystal inclusion which can account for up to 25\% of the dry weight of the sporulated cells. The amount of crystal protein produced by a B. thuringiensis culture in laboratory conditions (about 0.5 mg of protein per ml) and the size of the crystals (24) indicate that each cell has to synthesize $10^6$ to $2 \times 10^6$ $\delta$-endotoxin molecules during the stationary phase to form a crystal} \cite[p. 1]{Agaisse1995}.
\end{citacao}

Exemplo de nota de rodapé\footnote{Essa é uma nota de rodapé!}.

% ---


\end{document}

