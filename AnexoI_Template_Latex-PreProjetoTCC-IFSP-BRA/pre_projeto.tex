% Adaptado para IFSP-BRA por Profa. Dra. Ana Paula Müller Giancoli - 27/03/2020
%\renewcommand{\headrulewidth}{0pt} % retirando linha
%\renewcommand{\footrulewidth}{1pt} % colocando linha

%%%%%classe do documento%%%%%
\documentclass[article,12pt, a4paper]{abntex2}
%%%%%%%%%%%%%%%%%%%%%%%%%%%%%
\usepackage{helvet} % Usa a fonte Helvet que é parecida com Arial
\renewcommand{\familydefault}{\sfdefault} % define como default.
%\usepackage{lmodern}			% Usa a fonte Latin Modern		
\usepackage[T1]{fontenc}		% Selecao de codigos de fonte.
\usepackage[utf8]{inputenc}		% Codificacao do documento (conversão automática dos acentos)

%%%%%%%Pacotes%%%%%%%%%%%%%%
%\usepackage{lmodern}	
%\usepackage[utf8]{inputenc}
\usepackage[brazil]{babel}
\usepackage{graphicx}
\usepackage[table]{xcolor}
\usepackage{float}
\usepackage[brazilian,hyperpageref]{backref}	 % Paginas com as citações na bibl
\usepackage[alf,abnt-emphasize=bf]{abntex2cite}
%\citebrackets[]
\usepackage{indentfirst}
\usepackage{lipsum}
\usepackage{cmap}
\usepackage{capa}
\usepackage{fancyhdr}
\usepackage{fancybox} % Adicionado para criar uma moldura ao redor da página
\fancypage{\setlength{\fboxsep}{12pt}\fbox}{} % incluído a moldura da página

%%%%%%%%%%%%%%%%% cor das referências %%%%%%%%%%%%%%%%%
\hypersetup{
		colorlinks=true,       		% false: boxed links; true: colored links
		linkcolor=blue,        		% color of internal links
		citecolor=blue,        		% color of links to bibliography
		filecolor=magenta,     		% color of file links
		urlcolor=blue}

\IfFileExists{html.sty}
%%%%%%%%%%%%%%%%%%%%%%%%%%%%%%%%%%%%%%%%%%%%%%%%%%%%%%%%%%%%%%

\usepackage{url6023} % Adequado para atender a norma ABNT 6023:2018 por Profa. Dra. Ana Paula Müller Giancoli - 27/03/2020


%%%%%%%%%%%%% Linhas para Inserir o programa %%%%%%%%%%%%%%%%%%%%%
%Para trocar os dados da capa basta ir no arquivo capa.sty,alterando
%as seguintes linhas:

%41- \ABNTEXchapterfont\large \textbf{Área de concentração:} Sua Área de Concentração\\%area de concentracao

%42- \ABNTEXchapterfont\large \textbf{Linha de pesquisa:} Sua linha de pesquisa\\%linha de pesquisa

%43-  \ABNTEXchapterfont\large \textbf{Tema:} Seu tema\\%tema 

%O arquivo está dentro do projeto
%%%%%%%%%%%%%%%%%%%%%%%%%%%%%%%%%%%%%%%%%%%%%%%%%%%%%%%%%%%%%%%%%%



%%%%%%%%%%%%% tirando a impressão frente e verso  %%%%%%%%%%%%%%
\setboolean{@twoside}{false}
%%%%%%%%%%%%%%%%%%%%%%%%%%%%%%%%%%%%%%%%%%%%%%%%%%%%%%%%%%%%%%%%

%%%%%%%%%%%%%%%%%%%% Dados gerais %%%%%%%%%%%%%%%%%%%%%%%%%%%%%%
\titulo{ANEXO I\\Proposta de Trabalho de Conclusão de Curso (TCC)}
\autor{Prontuário: BPXXXXXX - Autor 1\\Prontuário: BPXXXXXX - Autor 2\\Prontuário: BPXXXXXX - Autor 3}
\local{Bragança Paulista}
\data{Ano}
\orientador{Prof. Dr(a)./MsC./Me. Orientador(a)}
\coorientador{Prof. Dr(a)./MsC./Me. Orientador(a)}
%%%%%%%%%%%%%%%%%%%%%%%%%%%%%%%%%%%%%%%%%%%%%%%%%%%%%%%%%%%%%%%%%

\begin{document}

%begin{titlepage}
	
%	\begin{center}
%		\includegraphics[width=70mm]{imagens/unb_bandeira.png}\\[0.2cm]
%		{\large Universidade de Brasília\\}
%		Faculdade de Tecnologia\\
%		Departamento de Engenharia Elétrica\\[2cm]
%		{\large \textbf{Pré-Projeto de Dissertação(Mestrado Acadêmico)}}\\[2cm]
%		{\large
%			Helton Alves de Azevedo\\[1cm]
%			\textbf{Área de concentração:} Sistemas Eletrônicos\\
%			\textbf{Linha de pesquisa:} Microeletrônica\\[0.5cm]
%			\textbf{Tema:} Construção e Estudo de Viabilidade de um Transceptor para o Padrão Ingenu RPMA Utilizando Circuitos Baseandos em CNTFETs e MOSFETs\\[2cm]
%			Orientador\\
%			Prof. Dr. Sandro Augusto Pavlik Haddad\\[0.5cm]
%			Brasília\\
%			2019\\
%			
%			}
%	\end{center}
%\end{titlepage}

\imprimircapa
\textual

%\thisfancypage{\setlength{\fboxsep}{12pt}\fbox}{} % incluído a moldura da página
\section{Título e Proposta do Projeto}

\subsection{Título do Projeto}
%\textbf{Indique aqui seu título / tema}
\begin{tabular}{|m{15cm}|}
			\hline
			\textbf{Indique o título do projeto aqui} \\
			\hline
		\end{tabular}
\subsection{Introdução}

Elaborar uma introdução ...

 

\subsection{Problema de pesquisa}
Neste item indique o problema a ser pesquisado. Pergunta.

\justify

\hspace{0.5cm} Por meio de citação de autores, contextualizar e justificar o porquê que o seu projeto é importante.
\subsection{Objetivo Geral}
Como objetivo geral ...

\subsubsection{Objetivos Específicos}
\begin{itemize}
	\item Objetivo 1;
	\item Objetivo 2;
	\item Objetivo n.
\end{itemize}


\subsection{Método de Pesquisa}
Definição da linha metodológica coerente com o problema.
Desenvolver ...

\justify

\hspace{0.5cm} Desenvolver ...

\subsection{Resultados esperados}
Descrever ...

\justify
Considerar os períodos 4, 5 e 6.
O Quadro~\ref{qua:cronograma} resume o cronograma de atividades da proposta de trabalho de conclusão de curso (TCC).


\begin{quadro}[H]
    \begin{center}
	    \caption{Cronograma de execução de atividades por semestre.}
		\label{qua:cronograma}
		\begin{tabular}{|m{3.5cm}|m{.2cm}|m{.2cm}|m{.2cm}|m{.2cm}|m{.2cm}|m{.2cm}|m{.2cm}|m{.2cm}|m{.2cm}|m{.2cm}|m{.2cm}|m{.2cm}|m{.2cm}|m{.2cm}|m{.2cm}|m{.2cm}|m{.2cm}|m{.2cm}|}
			\hline
			\centering
			\cellcolor{gray!50}\textbf{Entregáveis} & \multicolumn{18}{c|}{ \cellcolor{gray!50}\textbf{Atividades a serem realizadas / mês}}\\
			\hline
		    \centering
			\textbf{Anos} & \multicolumn{12}{c|}{\textbf{2022}} & \multicolumn{6}{c|}{\textbf{2023}} \\
			\hline
			\centering
            \cellcolor{gray!50}\textbf{Meses}&\cellcolor{gray!50}\textbf{1}&\cellcolor{gray!50}\textbf{2}&\cellcolor{gray!50}\textbf{3}&\cellcolor{gray!50}\textbf{4}&\cellcolor{gray!50}\textbf{5}&\cellcolor{gray!50}\textbf{6}&\cellcolor{gray!50}\textbf{7}&\cellcolor{gray!50}\textbf{8}&\cellcolor{gray!50}\textbf{9}&\cellcolor{gray!50}\textbf{10}&\cellcolor{gray!50}\textbf{11}&\cellcolor{gray!50}\textbf{12}&\cellcolor{gray!50}\textbf{1}&\cellcolor{gray!50}\textbf{2}&\cellcolor{gray!50}\textbf{3}&\cellcolor{gray!50}\textbf{4} &\cellcolor{gray!50}\textbf{5} &\cellcolor{gray!50}\textbf{6}\\
			\hline
			Formação do Grupo &\cellcolor{gray!50} & & & & & & & & & & & & & & & & & \\
			\hline
			Definição do tema &\cellcolor{gray!50} & & & & & & & & & & & & & & & & & \\
			\hline
			Introdução e problema &\cellcolor{gray!50} & & & & & & & & & & & & & & & & & \\
			\hline
			Justificativa e Objetivos & &\cellcolor{gray!50} & & & & & & & & & & & & & & & & \\
			\hline
			Pesquisa de metodologias & & &\cellcolor{gray!50} &\cellcolor{gray!50} & & & & & & & & & & & & & & \\
			\hline
			Resultados esperados & & & & &\cellcolor{gray!50} & & & & & & & & & & & & &\\
			\hline
			Cronograma e Apresentação & & & & &\cellcolor{gray!50} & & & & & & & & & & & & & \\
			\hline
			Revisão Bibliográfica & & & & & &  \cellcolor{gray!50} &\cellcolor{gray!50} &\cellcolor{gray!50} & \cellcolor{gray!50} & & & & & & & & & \\
			\hline
			Levantamento de Requisitos  & & & & & & &\cellcolor{gray!50} &\cellcolor{gray!50} &\cellcolor{gray!50}& & & & & & & & & \\
			\hline
			Elaborar Diagrama de Casos de uso e Classes  & & & & & & & & &\cellcolor{gray!50} &\cellcolor{gray!50} & & & & & & & & \\
			\hline
			Descrições dos Casos de Uso  & & & & & & & & &\cellcolor{gray!50} &\cellcolor{gray!50} & & & & & & & & \\
			\hline
			Elaboração dos Protótipos em ferramenta específica & & & & & & & & & &\cellcolor{gray!50} &\cellcolor{gray!50} &  \cellcolor{gray!50} & & & & & & \\
			\hline
			Entregas e Revisão & & & & & & & & & & & & & &\cellcolor{gray!50} &\cellcolor{gray!50} &\cellcolor{gray!50} & & \\ 
			\hline
			Desenvolvimento  & & & & & & &\cellcolor{gray!50} &\cellcolor{gray!50} &\cellcolor{gray!50} &\cellcolor{gray!50} &\cellcolor{gray!50} &\cellcolor{gray!50} &\cellcolor{gray!50} &\cellcolor{gray!50} &\cellcolor{gray!50} &\cellcolor{gray!50} &\cellcolor{gray!50} & \\
			\hline
			Finalização da documentação  & & & & & & & & & & & & & & & & &\cellcolor{gray!50} & \\
			\hline
			Defesa do TCC & & & & & & & & & & & & & & & & & &\cellcolor{gray!50} \\
			\hline
		\end{tabular}
		\vspace{0.5cm}	\\Fonte: Autoria própria (2023)
	\end{center}
\end{quadro}
%\thisfancypage{\setlength{\fboxsep}{12pt}\fbox}{} % incluído a moldura da página

	% ----------------------------------------------------------
	% ELEMENTOS PÓS-TEXTUAIS
	% ----------------------------------------------------------
    \newpage
  %  \thisfancypage{\setlength{\fboxsep}{12pt}\fbox}{}

    \bibliography{referencias}
\vspace{5cm}


    \begin{center}
		\begin{tabular}{|m{5cm}|m{10cm}|}
			\hline
			\textbf{Alunos} & \textbf{Assinatura} \\
			\hline
			\textit{Nome do Autor 1} & \\
			\hline
			\textit{Nome do Autor 2} & \\
			\hline
			\textit{Nome do Autor 3} & \\
			\hline
		\end{tabular}
	\end{center}

\vspace{0.5cm}

	\begin{center}
		\begin{tabular}{|m{5cm}|m{10cm}|}
			\hline
			\textbf{Orientador(es)} & \textbf{Assinatura} \\
			\hline
			\textit{Nome do Orientador 1} & \\
			\hline
			\textit{Nome do Coorientador 1} & Se não tiver, comentar a linha com percentual na frente e no hline seguinte também.\\
			\hline
		\end{tabular}
	\end{center}

%incluído somente com exemplos. Deve ser desconsiderado no documento final. Para isso, basta inserir um percentual na frente do \input. Igual foi feito em anexos.


\begin{apendicesenv}
	
	% Imprime uma página indicando o início dos apêndices
	\partapendices
	
	% ----------------------------------------------------------
	\chapter{Dicas de comandos LateX}
	% ----------------------------------------------------------
	

     Exemplos de comandos para texto e referências:
     
     \begin{itemize}
     	\item Para iniciar um novo parágrafo, basta deixar uma linha em branco no código fonte;
     	\item Não force o compilador a pular mais de uma linha, pois terá influência negativa na composição do documento;
     	\item Sempre deixe o \LaTeX\ realizar a formatação de parágrafos e posicionamento de elementos;
     	\item Utilização de aspas simples (abertura \verb|`|, fechamento \verb|'|): `Texto entre aspas simples';
     	\item Utilização de aspas duplas (abertura \verb|``|, fechamento \verb|''|): ``Texto entre aspas duplas'';
     	\item Negrito (comando \verb|\textbf|): \textbf{texto em negrito};
     	\item Itálico (comando \verb|\textit|): \textit{texto em itálico};
     	\item Sublinhado (comando \verb|\underline|): \underline{texto sublinhado};
     	\item Negrito e itálico (usar comandos juntos): \textbf{\textit{texto em negrito e itálico}};
     	\item Alterar cor do texto (comando \verb|\textcolor{cor}{texto}|):
     	\begin{itemize}
     		\item Exemplo \verb|\textcolor{red}{texto}|: \textcolor{red}{texto vermelho};
     		\item Exemplo \verb|\textcolor[RGB]{255, 102, 0}|: \textcolor[RGB]{255, 102, 0}{texto laranja};
     		\item Exemplo \verb|\textcolor[HTML]{006AD7}|: \textcolor[HTML]{006AD7}{texto azul};
     	\end{itemize}
     	\item Ambiente matemático inline (comando \verb|$ expressão $|): $s = x^2-2x +1$;
     	\item Referência normal (comando \verb|\cite|):
     	\begin{itemize}
     		\item \cite{Agaisse1995};
     		\item \cite{Abedi2014};
     		\item \cite{Nelson2014};
     	\end{itemize}
     	\item Referência normal com mais de uma obra (comando \verb|\cite|):
     	\begin{itemize}
     		\item \cite{Abedi2014, Agaisse1995};
             \item \cite{AgapitoTenfen2014, Abedi2014, Nelson2014};
     	\end{itemize}
     	\item Referência nome e ano (comando \verb|\citeaonline|):
     	\begin{itemize}
     		\item \citeonline{Agaisse1995};
     		\item \citeonline{Abedi2014};
     		\item \citeonline{Nelson2014};
     		\item \citeonline{Baum2016}
     	\end{itemize}
     \end{itemize}
     
\end{apendicesenv} %incluído somente com exemplos. Deve ser desconsiderado no documento final. Para isso, basta inserir um percentual na frente do \input. Igual foi feito em anexos.

%% Texto ou documento não elaborado pelo autor, que serve de fundamentação, comprovação e ilustração.

% ---
% Inicia os anexos
% ---
\begin{anexosenv}
	
	% Imprime uma página indicando o início dos anexos
	\partanexos
	
	% ----------------------------------------------------------
	\chapter{Título do Anexo A}
	% ----------------------------------------------------------
	
	Texto do Anexo A.
	

	
	
\end{anexosenv}

\end{document}

