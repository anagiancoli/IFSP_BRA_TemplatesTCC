\chapter{Exemplos}
\label{cap:99}

Texto considerando a revisão da literatura pertinente, dividido em seções e subseções.

Este é um exemplo de como usar figuras. Referência cruzada: Figura~\ref{fig:exemplo}

\FloatBarrier
\begin{figure}[!htbp]
	\centering
	\caption{Exemplo de figura}
	%scale redimensiona a figura.
	%1.5 = 150% do tamanho original
	%1 = 100% do tamanho original
	%0.20 = 20% do tamanho original
	\fbox{\includegraphics[scale=1]{imagens/IFSP-BRA.png}}
	\legend{Fonte: Disponível em: http://bra.ifsp.edu.br. Acesso em: 27 mar. 2020}
        %\legend{Fonte: Autor (ano) ou \citetext{chave}}
	\label{fig:exemplo}
\end{figure}
\FloatBarrier


Este é um exemplo de como usar tabelas. Referência cruzada: Tabela~\ref{tab:exemplo}

\FloatBarrier
\begin{table}[!htbp]
\centering
\caption{Exemplo de tabela de 2 colunas}
	\begin{tabular}{ c | c }
		\hline
		\textbf{Coluna 1} & \textbf{Coluna 2} \\ \hline
		Dado 1a           & Dado 2a           \\ \hline
		Dado 1b           & Dado 2b           \\ \hline
		Dado 1c           & Dado 2c           \\ \hline
		Dado 1d           & Dado 2d           \\ \hline
	\end{tabular}
	\\ \vspace{0.2cm}
	Fonte: Autoria própria (ano)
	\label{tab:exemplo}
\end{table}
\FloatBarrier


Este é um exemplo de como usar quadros. Referência cruzada: Quadro~\ref{qua:exemplo}

\FloatBarrier
\begin{quadro}[!htbp]
	\centering
	\caption{Exemplo de quadro}
	\includegraphics[scale=.7]{imagens/exemploQuadro}
	\\Fonte: Autoria própria (ano)
	\label{qua:exemplo}
\end{quadro}
\FloatBarrier

Este é um exemplo de como usar quadros. Referência cruzada: Quadro~\ref{tab:exemploquad}

\FloatBarrier
\begin{quadro}[!htbp]
\centering
\caption{Exemplo de Quadro de 3 colunas}
	\begin{tabular}{ | m{10em} | m{4cm}| m{4cm} | }
		\hline
		\textbf{Coluna 1} & \textbf{Coluna 2} & \textbf{Coluna 3} \\ \hline
		Dado 1a           & Dado 2a & \\ \hline
		Dado 1b           & Dado 2b & \\ \hline
		Dado 1c           & Dado 2c & \\ \hline
		Dado 1d           & Dado 2d & \\ \hline
	\end{tabular}
	\\ \vspace{0.2cm}
	Fonte: Autoria própria (ano)
	\label{tab:exemploquad}
\end{quadro}
\FloatBarrier

Este é um exemplo de como usar quadros. Referência cruzada: Quadro~\ref{tab:exemplo2}

\FloatBarrier
\begin{quadro}[!htbp]
\centering
\caption{Exemplo de Quadro de 2 colunas}
	\begin{tabular}{ | m{10em} | m{4cm}| }
		\hline
		\textbf{Coluna 1} & \textbf{Coluna 2}  \\ \hline
		Dado 1a           & Dado 2a  \\ \hline
		Dado 1b           & Dado 2b  \\ \hline
		Dado 1c           & Dado 2c  \\ \hline
		Dado 1d           & Dado 2d  \\ \hline
	\end{tabular}
	\\ \vspace{0.2cm}
	Fonte: Autoria própria (ano)
	\label{tab:exemplo2}
\end{quadro}
\FloatBarrier

Este é um exemplo de como usar equações. Referência cruzada: Equação~\ref{eq:exemplo}

\begin{equation}
\sum_{i=1}^{n} i = \frac{n(n+1)}{2}
\label{eq:exemplo}
\end{equation}

\clearpage

Exemplo de inserção de lista de código fonte (\textbf{\textcolor{red}{não use acentos no código!}}):

\lstinputlisting[language=Java]{fontes/ClasseExemplo.java} 



Este é um exemplo de como inserir texto sem formatação (ambiente verbatim):

\begin{verbatim}
	Texto sem formatação, como espaçamento igual.
\end{verbatim}


Exemplo de lista de itens:

\begin{itemize}
	\item \textbf{Item 1:} texto...;
	\item \textbf{Item 2:} texto...;
    \begin{itemize}
            \item \textbf{Subitem:} texto...;
            \item \textbf{Subitem:} texto...;
            \item \textbf{Subitem:} texto...;
        \end{itemize}
	\item \textbf{Item 3:} texto...;
	\item \textbf{Item n:} texto....
\end{itemize}


Exemplo de lista numerada:

\begin{enumerate}
	\item \textbf{Item:} texto...;
	\item \textbf{Item:} texto...;
    \begin{enumerate}
        \item \textbf{Subitem:} texto...;
        \item \textbf{Subitem:} texto...;
        \item \textbf{Subitem:} texto...;
    \end{enumerate}
	\item \textbf{Item:} texto...;
	\item \textbf{Item:} texto....
\end{enumerate}

Tipos de referência a serem utilizados no arquivo referencias.bib:
Exemplos de estruturas.\\

@article{<citation key>,
    author        = {},
    title         = {},
    journaltitle  = {},
    year          = {}
}

@online{<citation key>,
    author        = {},
    title         = {},
    year          = {},
    url           = {},
    urlaccessdate = {}
}

@book{<citation key>,
    author        = {},
    title         = {},
    year          = {}
}

@misc{<citation key>,
    author        = {},
    title         = {},
    year          = {},
    url           = {},
    urlaccessdate = {}
}


Exemplos de comandos para texto e referências:

\begin{itemize}
	\item Para iniciar um novo parágrafo, basta deixar uma linha em branco no código fonte;
	\item Não force o compilador a pular mais de uma linha, pois terá influência negativa na composição do documento;
	\item Sempre deixe o \LaTeX\ realizar a formatação de parágrafos e posicionamento de elementos;
	\item Utilização de aspas simples (abertura \verb|`|, fechamento \verb|'|): `Texto entre aspas simples';
	\item Utilização de aspas duplas (abertura \verb|``|, fechamento \verb|''|): ``Texto entre aspas duplas'';
	\item Negrito (comando \verb|\textbf|): \textbf{texto em negrito};
	\item Itálico (comando \verb|\textit|): \textit{texto em itálico};
	\item Sublinhado (comando \verb|\underline|): \underline{texto sublinhado};
	\item Negrito e itálico (usar comandos juntos): \textbf{\textit{texto em negrito e itálico}};
	\item Alterar cor do texto (comando \verb|\textcolor{cor}{texto}|):
	\begin{itemize}
		\item Exemplo \verb|\textcolor{red}{texto}|: \textcolor{red}{texto vermelho};
		\item Exemplo \verb|\textcolor[RGB]{255, 102, 0}|: \textcolor[RGB]{255, 102, 0}{texto laranja};
		\item Exemplo \verb|\textcolor[HTML]{006AD7}|: \textcolor[HTML]{006AD7}{texto azul};
	\end{itemize}
	\item Ambiente matemático inline (comando \verb|$ expressão $|): $s = x^2-2x +1$;
	\item Referência normal (comando \verb|\cite|):
	\begin{itemize}
		\item \cite{Agaisse1995};
		\item \cite{Abedi2014};
		\item \cite{Baum2016};
        
	\end{itemize}
	\item Referência normal com mais de uma obra (comando \verb|\cite|):
	\begin{itemize}
		\item \cite{Abedi2014, Agaisse1995};
		\item \cite{AgapitoTenfen2014, Baum2016, Nelson2014};
	\end{itemize}
	\item Referência nome e ano (comando \verb|\citeauthorandyear|):
	\begin{itemize}
		\item \citeauthorandyear{bervian2007a};
		\item \citeonline{bervian2007a};
		\item \citeauthorandyear{documento2018};
		\item \citeauthorandyear{Abedi2014};
        \item \citeonline{schaefer2022};
        \item \citeonline{castro2019};
	\end{itemize}
	\item Referência apud ({mais antigo} {mais novo}):\\
	\verb|\apud[pagina]{autor-indireto}{autor-direto}|):
	\begin{itemize}
		\item \apud{Agaisse1995}{Abedi2014};
		\item \apudonline{Agaisse1995}{Abedi2014};
	\end{itemize}
	\item Referência apud [pagina]({mais antigo} {mais novo}):\\
	\verb|\apud[pagina]{autor-indireto}{autor-direto}|):
	\begin{itemize}
		\item \apud[p. 81]{Agaisse1995}{Abedi2014};
		\item \apudonline[p. 81]{Agaisse1995}{Abedi2014};
	\end{itemize}
	
	\item Referência do mesmo autor, mesmo ano com obras distintas 
	\verb|\citeauthorandyear|):
	\begin{itemize}
		\item \citeauthorandyear{Agaisse1996a};
		\item \citeauthorandyear{Agaisse1996b};
		\item \cite{Agaisse1996a};
		\item \cite{Agaisse1996b};
	\end{itemize}
	
	
\end{itemize}


Exemplo 1 de citação direta:

\begin{citacao}
	Os 20 aminoácidos usualmente encontrados como resíduos em proteínas contém um grupo $\alpha$-carboxil, um grupo $\alpha$-amino e um grupo R distinto substituído no átomo de carbono $\alpha$. O átomo de carbono $\alpha$ de todos os aminoácidos, com exceção da glicina, é assimétrico e, portanto, os aminoácidos podem existir em pelo menos duas formas estereoisoméricas. Somente os estereoisômeros L, com uma configuração relacionada à configuração absoluta da molécula de referência L-gliceraldeído, são encontrados em proteínas \cite[p. 81]{Nelson2014}.
\end{citacao}

Exemplo 2 de citação direta:

\begin{citacao}
	\textit{These various insecticidal proteins are synthesized during the stationary phase and accumulate in the mother cell as a crystal inclusion which can account for up to 25\% of the dry weight of the sporulated cells. The amount of crystal protein produced by a B. thuringiensis culture in laboratory conditions (about 0.5 mg of protein per ml) and the size of the crystals (24) indicate that each cell has to synthesize $10^6$ to $2 \times 10^6$ $\delta$-endotoxin molecules during the stationary phase to form a crystal} \cite[p. 1]{Agaisse1995}.
\end{citacao}

Exemplo de nota de rodapé\footnote{Essa é uma nota de rodapé!}.

Exemplo de rodapé com referência indicada nele\footnote{\citetext{marciano2020}}
