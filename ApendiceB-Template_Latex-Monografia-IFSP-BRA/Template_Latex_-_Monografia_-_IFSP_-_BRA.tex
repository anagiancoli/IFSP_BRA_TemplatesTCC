% Desenvolvido por: Prof. Dr. David Buzatto
%
% Baseado na documentação do abntex2 e nos modelos em
% Microsoft Word propostos pela Profa. Dra. Rosana F. L. Rodrigues
% e pela bibliotecária M.Sc. Maria Carolina Gonçalves do câmpus
% São João da Boa Vista do IFSP.
%
% Versão 1.51
% Data: 06/11/2018
% Modelo adaptado para IFSP-BRA por Profa. Dra. Ana Paula Müller Giancoli - 27/03/2020
\PassOptionsToPackage{unicode}{hyperref}
\PassOptionsToPackage{naturalnames}{hyperref}
\documentclass[
	% -- opções da classe memoir --
	12pt,				% tamanho da fonte
	openright,			% capítulos começam em pág ímpar (insere página vazia caso preciso)
	oneside,			% para impressão em verso e anverso. Oposto a oneside, twoside
	a4paper,			% tamanho do papel. 
	%normalfigtabnum,
	%pnumromarab,
	% -- opções da classe abntex2 --
	%chapter=TITLE,		% títulos de capítulos convertidos em letras maiúsculas
	%section=TITLE,		% títulos de seções convertidos em letras maiúsculas
	%subsection=TITLE,	% títulos de subseções convertidos em letras maiúsculas
	%subsubsection=TITLE,% títulos de subsubseções convertidos em letras maiúsculas
	% -- opções do pacote babel --
	english,			% idioma adicional para hifenização
	french,				% idioma adicional para hifenização
	spanish,			% idioma adicional para hifenização
	brazil,				% o último idioma é o principal do documento
]{abntex2}

% ---
% Pacotes básicos 
% ---
\usepackage{helvet} % Usa a fonte Helvet que é parecida com Arial
\renewcommand{\familydefault}{\sfdefault} % define como default.
%\usepackage{lmodern}			% Usa a fonte Latin Modern		
\usepackage[T1]{fontenc}		% Selecao de codigos de fonte.
\usepackage[utf8]{inputenc}		% Codificacao do documento (conversão automática dos acentos)
\usepackage{lastpage}			% Usado pela Ficha catalográfica
\usepackage{indentfirst}		% Indenta o primeiro parágrafo de cada seção.
\usepackage{color, xcolor, colortbl}	% Controle das cores
\usepackage{graphicx}			% Inclusão de gráficos
\usepackage{microtype} 			% para melhorias de justificação
\usepackage{hyperref}
\usepackage{subfig}
\usepackage{epigraph}
\usepackage{placeins}
\usepackage{multirow}
\usepackage[figuresright]{rotating}
\usepackage{chemfig}
\usepackage{amsmath}
\usepackage{amssymb}
\usepackage{enumitem}
\usepackage{bigints}
\usepackage{listings}
\usepackage{etoolbox}
\usepackage[final]{pdfpages} % inserir pdf
\usepackage{bigstrut}
% ---
% Pacotes adicionais, usados apenas no âmbito do Modelo Canônico do abnteX2
% ---
\usepackage{lipsum}				% para geração de dummy text
% ---


\newcommand{\Csh}{C{\#}}


% ---
% Reconfiguração o tamanho da fonte dos capítulos
% ---
\captionstyle{\raggedright} % (para as legendas; use \legend para fonte)
% Fontes padrao de part, chapter, section, subsection e subsubsection
\renewcommand{\ABNTEXchapterfontsize}{\bfseries\Large}
\renewcommand{\ABNTEXpartfont}{\ABNTEXchapterfont}
\renewcommand{\ABNTEXpartfontsize}{\ABNTEXchapterfontsize}
\renewcommand{\ABNTEXsectionfont}{\ABNTEXchapterfont}
\renewcommand{\ABNTEXsectionfontsize}{\bfseries\large}
\renewcommand{\ABNTEXsubsectionfont}{\ABNTEXsectionfont}
\renewcommand{\ABNTEXsubsectionfontsize}{\large}
\renewcommand{\ABNTEXsubsubsectionfont}{\ABNTEXsubsectionfont}
\renewcommand{\ABNTEXsubsubsectionfontsize}{\large}
\renewcommand{\ABNTEXsubsubsubsectionfont}{\ABNTEXsubsectionfont}
\renewcommand{\ABNTEXsubsubsubsectionfontsize}{\large}
% ---
% Pacotes de citações
% ---
\usepackage[brazilian,hyperpageref]{backref}	 % Paginas com as citações na bibl
\usepackage[alf,abnt-emphasize=bf,abnt-etal-text=emph]{abntex2cite}  % Citações padrão ABNT
\usepackage{url6023} % Adequado para atender a norma ABNT 6023:2018 por Profa. Dra. Ana Paula Müller Giancoli - 27/03/2020
% ---
% Listagens
% ---
\definecolor{corComentario}{RGB}{150,150,150}
\definecolor{corString}{RGB}{206,123,0}
\definecolor{corPalavraChave}{RGB}{0,0,230}
\lstset{
	numbers=left,
	stepnumber=1,
	firstnumber=1,
	numberstyle=\footnotesize,
	extendedchars=true,
	breaklines=true,
	lineskip=0pt,
	frame=tb,
	basicstyle=\ttfamily\footnotesize,
	showstringspaces=false,
	stringstyle=\color{corString},
	commentstyle=\color{corComentario},
	keywordstyle=\color{corPalavraChave}
}

% ---
% Definições de variáveis do documento
% ---
\newcolumntype{Y}{>{\centering\arraybackslash}X}
% Ano
\newcommand{\ano}[1]{\def \oano {#1}}
\newcommand{\imprimirano}{\oano}
% Mês
\newcommand{\mes}[1]{\def \omes {#1}}
\newcommand{\imprimirmes}{\omes}
% Subtítulo
\newcommand{\subtitulo}[1]{\def \osubtitulo {#1}}
\newcommand{\imprimirsubtitulo}{\osubtitulo}
% Área de Atuação
\newcommand{\area}[1]{\def \aarea {#1}}
\newcommand{\imprimirarea}{\aarea}
% Coorientador
\renewcommand{\coorientador}[1]{\def \ocoorientador {#1}}
\renewcommand{\imprimircoorientador}{\ocoorientador}
% Grau
\newcommand{\grau}[1]{\def \ograu {#1}}
\newcommand{\imprimirgrau}{\ograu}
% Curso
\newcommand{\curso}[1]{\def \ocurso {#1}}
\newcommand{\imprimircurso}{\ocurso}

% --
% Utilize a instrução \relax [...] para iniciar as frases que precisam de colchetes e reticências ou \text{[ \dots ]}. Além do pacote: \usepackage{amsmath, amsfonts, amssymb} % importado para usar  \text{[ \dots ] }
% --


% ---
% Informações de dados para CAPA e FOLHA DE ROSTO
% ---
\ano{ANO}
\mes{MÊS}
\local{Bragança Paulista}
\area{Computação}
% ---
% Informações do Curso, grau, coorientador, orientador
% ---
\orientador{Prof./Profa. Me./MsC. /Dr./Dra. Nome Completo}
% caso não haja coorientador, comente a linha abaixo
\coorientador{Prof./Profa. Me./MsC./Dr./Dra. Nome Completo}
\grau{Tecnólogo em Análise e Desenvolvimento de Sistemas}
\curso{Tecnologia em Análise e Desenvolvimento de Sistemas}
% ---
% Informações do Título e subtítulo
% ---
\titulo{Título }
% caso não haja subtítulo, comente a linha abaixo usando o percentual na frente
\subtitulo{subtítulo (se houver, caso não exista, comente a linha com o percentual na frente)}
\tipotrabalho{Trabalho de Conclusão de Curso}
% --- 
% Informações dos autores e instituição
% ---
\autor{Seu Nome Completo}
\instituicao{Instituto Federal de Educação, Ciência e Tecnologia de São Paulo\\Câmpus Bragança Paulista}
% --- 
% Criando o preambulo
% ---
\preambulo{\imprimirtipotrabalho\ apresentado ao Instituto Federal de Educação, Ciência e Tecnologia de São Paulo, como parte dos requisitos para a obtenção do título de \imprimirgrau.
\\
Área de Concentração: \imprimirarea}
% ---
% ---
% Configurações de aparência do PDF final
% ---
% alterando o aspecto da cor azul
\definecolor{blue}{RGB}{41,5,195}
\makeatletter
\hypersetup{
	%pagebackref=true,
	pdftitle={\@title}, 
	pdfauthor={\@author},
	pdfsubject={\imprimirpreambulo},
	pdfcreator={Nome Completo},
	pdfkeywords={Palavra chave 1}{Palavra chave 2}{Palavra chave 3}{Palavra chave n}, 
	colorlinks=true,       		% false: boxed links; true: colored links
	linkcolor=black,          	% color of internal links
	citecolor=black,       		% color of links to bibliography
	filecolor=black,      		% color of file links
	urlcolor=black,
	bookmarksdepth=4
}
\makeatother
% ---
% Comandos do autor
% ---
% comando para inserir autor e ano
\newcommand{\citeauthorandyear}[1]{\citeauthoronline{#1} (\citeyear{#1})}
% ---
% Novo list of (listings) para Quadros
% ---
\newcommand{\quadroname}{Quadro}
\newcommand{\listofquadrosname}{Lista de quadros}
\newfloat[chapter]{quadro}{loq}{\quadroname}
\newlistof{listofquadros}{loq}{\listofquadrosname}
\newlistentry{quadro}{loq}{0}
% ---
% configurações para atender às regras da ABNT
% ---
\setfloatadjustment{quadro}{\centering}
\counterwithout{quadro}{chapter}
\renewcommand{\cftquadroname}{\quadroname\space} 
\renewcommand*{\cftquadroaftersnum}{\hfill--\hfill}
% ---
% Configuração de posicionamento padrão:
% ---
\setfloatlocations{quadro}{hbtp}
% --- 
% Espaçamentos entre linhas e parágrafos 
% --- 
% O tamanho do parágrafo é dado por:
\setlength{\parindent}{1.3cm}
% Controle do espaçamento entre um parágrafo e outro:
\setlength{\parskip}{0.2cm}  % tente também \onelineskip
% ---
% compila o indice
% ---
\makeindex
% ---
% ---------------------------------------------------------------------------------
%                                   INÍCIO DO DOCUMENTO
% ---------------------------------------------------------------------------------
\begin{document}
\selectlanguage{brazil}
% Retira espaço extra obsoleto entre as frases.
\frenchspacing 
% --- 
% Elementos Pré Textuais 
% ---
% \pretextual
% ---
% Capa
% ---
%\imprimircapa
% capa personalizada

\begin{center}
	
	%\center
	\ABNTEXchapterfont\large{\imprimirinstituicao} 
	\vspace{3.5cm}
	
    \ABNTEXchapterfont\large{\imprimirautor}
	\vspace{3.5cm}
	
    \ABNTEXchapterfont\bfseries\large{\imprimirtitulo\ifdef{\osubtitulo}{}{}}
    
    \ifdef{\osubtitulo}{\ABNTEXchapterfont\normalsize\imprimirsubtitulo}{}
	\vfill
	
	\normalsize\mdseries{\imprimirlocal}
	
	\normalsize{\imprimirano}
	
	\vspace*{2cm}
	
\end{center}



% ---
% Folha de rosto
% (o * indica que haverá a ficha bibliográfica)
% ---
%\imprimirfolhaderosto*
\begin{center}
   	
   	\ABNTEXchapterfont\large{\imprimirautor}
   	\vspace{2.5cm}
   	
    \ABNTEXchapterfont\bfseries\large\imprimirtitulo\ifdef{\osubtitulo}{}{}
                           
    \ifdef{\osubtitulo}{\ABNTEXchapterfont\normalsize\imprimirsubtitulo}{}
   	\vspace{2.5cm}
   	   	
   	\hspace{.4\textwidth}
   	\begin{minipage}{.5\textwidth}
   		\SingleSpacing
   		\mdseries\normalsize\imprimirpreambulo
   		
   		\vspace{\onelineskip}
   		
   		Orientador: \imprimirorientador
   		
        \ifdef{\ocoorientador}{
     		\vspace{\onelineskip}
   		
    %		Coorientador: \imprimircoorientador
        }{}
   		
   	\end{minipage}%
    \vfill
   	
   	\mdseries\normalsize{\imprimirlocal}
   	
   	\normalsize{\imprimirano}
   	
   	\vspace*{2cm}
   	
\end{center}

% ---
% Inserir a ficha catalográfica
% ---
%
% Este é um exemplo de ficha catalográfica.
%
% A ficha catalográfica final deve ser requisitada pelo aluno no site da biblioteca e inserida neste documento para a entrega da versão final. (ficha catalográfica gerada pela Biblioteca do IFSP-BRA após a defesa e para a entrega do Trabalho Final (TCC) corrigido). Vide Regulamento de TCC.
%
% Modelo de Ficha catalográfica adequado para IFSP-BRA por Profa. Dra. Ana Paula Müller Giancoli
%


\vspace{15cm}

\begin{fichacatalografica}
	\sffamily
	\vspace*{\fill}					% Posição vertical
	\begin{center}					% Minipage Centralizado
	\fbox{
	    \begin{minipage}[c][8cm]{13.5cm}	% Largura
	        \small \imprimirautor % Sobrenome, Nome do autor
	
	    \hspace{0.5cm} \imprimirtitulo  / \imprimirautor. --
	    \imprimirlocal,  27 de março de 2020-
	
	    \hspace{0.5cm} \thelastpage p. : il. (algumas color.) ; 30 cm.\\
	   
	    \hspace{0.5cm} \imprimirorientadorRotulo~\imprimirorientador\\
	
	    \hspace{0.5cm}
	    \parbox[t]{\textwidth}{\imprimirtipotrabalho~--~\\ \imprimirinstituicao, 27 de março de 2020.
	    }\\
	
	    \hspace{0.5cm}
		1. Palavra-chave1.
		2. Palavra-chave2.
		2. Palavra-chave3.
		I. Orientador.\\
		II. Universidade xxx.
		III. Faculdade de xxx.
		IV. Título 			
	    \end{minipage}
	  }
	\end{center}
\end{fichacatalografica}

% ---
% Inserir folha de aprovação
% ---
% ---
% Este é um exemplo de folha de aprovação.
% ---
% Modelo de Folha de Aprovação criado e adequado para IFSP-BRA por Profa. Dra. Ana Paula Müller Giancoli - 27/03/2020
% ---

\thispagestyle{empty}
\clearpage
\begin{center}
   	
   	\ABNTEXchapterfont\large{\imprimirautor}
   	\vspace{1cm}
   	
    \ABNTEXchapterfont\bfseries\large\imprimirtitulo\ifdef{\osubtitulo}{}{}
                           
    \ifdef{\osubtitulo}{\ABNTEXchapterfont\normalsize\imprimirsubtitulo}{}
   	\vspace{0.5cm}
   	   	
   	\hspace{.4\textwidth}
   	\begin{minipage}{.5\textwidth}
   		\SingleSpacing
   		\mdseries \normalsize\imprimirpreambulo
   	 
   	\end{minipage}
   	
   	\vspace{1cm}
   	\SingleSpacing \mdseries \normalsize Aprovado pela Banca Examinadora em {\imprimirlocal},  \rule{1cm}{0.2pt} \hspace{0.01cm} de  \rule{5cm}{0.2pt} \hspace{0.01cm} de 2020.
    
    \vspace{1cm}
    \rule{10cm}{0.2pt}
    {\\Prof(a). Dr(a)./MsC. Nome Completo Orientador \\}
    {\footnotesize{Instituição do orientador}}
    
    \vspace{0.75cm}
     \rule{10cm}{0.2pt}
    {\\Prof(a). Dr(a)./MsC. Nome Completo Banca 2 \\}
    {\footnotesize{Instituição do membro da banca 2 }}
    
    \vspace{0.75cm}
     \rule{10cm}{0.2pt}
    {\\Prof(a). Dr(a)./MsC. Nome Completo Banca 3 \\}
    {\footnotesize{Instituição do membro da banca 3 }}
    
\end{center}

% ---
% Dedicatória
% ---
\input{pre05Dedicatoria}

% ---
% Agradecimentos
% ---
\input{pre06Agradecimentos}

% ---
% Epígrafe
% ---
\input{pre07Epigrafe}

% ---
% Resumos
% ---
\input{pre08Resumo}
\setlength{\absparsep}{18pt} % ajusta o espaçamento dos parágrafos do resumo
\begin{resumo}[Abstract]
	
	\begin{otherlanguage*}{english}
		
		Elemento obrigatório. É a versão do resumo em português para o idioma de divulgação internacional. Deve ser antecedido pela referência do estudo.
		
		\vspace{\onelineskip}
		 
		\textbf{Keywords}: Keyword 1. Keyword 2. Keyword 3. Keyword n.
		
	\end{otherlanguage*}

\end{resumo} 

% ---
% inserir lista de ilustrações
% ---
\pdfbookmark[0]{\listfigurename}{lof}
\listoffigures*
\cleardoublepage

% ---
% inserir lista de quadros
% ---
\pdfbookmark[0]{\listofquadrosname}{loq}
\listofquadros*
\cleardoublepage

% ---
% inserir lista de tabelas
% ---
\pdfbookmark[0]{\listtablename}{lot}
\listoftables*
\cleardoublepage

% ---
% inserir lista de abreviaturas e siglas
% ---
\begin{siglas}
	\item[1D] Uma dimensão
	\item[2D] Duas dimensões
	\item[3D] Três dimensões
\end{siglas}

% ---
% inserir lista de símbolos
% ---
\begin{simbolos}
	\item[$\alpha$] Letra grega minúscula Alfa
	\item[$\beta$] Letra grega minúscula Beta
\end{simbolos}

% ---
% inserir o sumário
% ---
\pdfbookmark[0]{\contentsname}{toc}
\tableofcontents*
\cleardoublepage

% --- 
% Elementos Textuais 
% ---
\textual
\chapter{Introdução}
\label{cap:01}

O objetivo deste documento é esclarecer aos autores o formato que deve ser utilizado nos relatórios técnicos a serem submetidos ao final dos cursos de Graduação e Pós-Graduação do IFSP câmpus Bragança Paulista. Este documento está escrito de acordo com o modelo indicado para a formatação dos relatórios técnicos; assim, serve de referência, ao mesmo tempo em que comenta os diversos aspectos da formatação.

Observe as instruções e formate seu relatório técnico de acordo com este padrão. Lembre-se que uma formatação correta contribui para uma boa avaliação do seu trabalho.

Além disso, neste documento estão listadas as seções obrigatórias que você deverá fornecer, bem como os exemplos dos comandos mais comuns que serão utilizados na construção de seu documento. Para pesquisar sobre mais comandos, recomenda-se a utilização do site \url{https://ctan.org/}, que é a biblioteca principal do \LaTeX, e o do site \url{https://tex.stackexchange.com} que é uma das principais comunidades para solução de dúvidas relacionadas a \LaTeX. Ambas são em inglês.

A introdução é um elemento preliminar, utilizado para fornecer informações específicas, comentar tecnicamente o conteúdo do trabalho, além de evidenciar as motivações que levaram o autor à escolha de determinado tema.

Trata-se de importante estratégia de aproximação, pois permite valorizar a escolha do assunto, mostrar a relevância da abordagem temática e esclarecer quanto ao passo-a-passo utilizado na estruturação do texto.

Na introdução, o leitor terá condições de avaliar:

\begin{itemize}
	\item O grau de informação, conhecimento e competência técnica do autor relativamente ao assunto a ser tratado;
	\item A qualidade, a eficiência, a originalidade e o ineditismo de sua abordagem;
	\item A pertinência das informações apresentadas e a possibilidade de acrescentar algo de novo ao universo conceitual do leitor.
\end{itemize}


\section{Objetivos}

\subsection{Objetivo Geral}

Qual seu objetivo geral.

\subsection{Objetivos Específicos}
\begin{itemize}
	\item Objetivo específico 1;
	\item Objetivo específico 2;
	\item Objetivo específico n.
\end{itemize}


\section{Participantes do projeto e seus papéis}
Descrever...

\section{Problemas, dificuldades ou obstáculos encontrados/enfrentados}
Descrever...

\section{Aprendizado e experiência profissional obtidos}
Descrever...


\chapter{Fundamentação Teórica}
\label{cap:02}

\textit{Neste capítulo, aborda-se conceitos e trabalhos da literatura relacionados ao tema principal do trabalho, que visam contextualizar os conhecimentos que serão mostrados em capítulos subsequentes, possibilitando uma melhor compreensão do assunto tratado de forma simples e objetiva.} \\


\chapter{Desenvolvimento}
\label{cap:03}

\textit{Este capítulo tem por objetivo apresentar a caracterização de pesquisa científica, a metodologia empregada bem como a arquitetura física e lógica do projeto.}\\

\section{Caracterização da pesquisa}

\subsection{Instrumentos de coleta}

\subsection{Participantes}

\section{Arquitetura Física}

\section{Arquitetura Lógica}

\section{Protótipos}
\chapter{Resultados}
\label{cap:04}

\textit{Este capítulo apresenta os resultados finais obtidos na conclusão deste trabalho. Para uma maior compreensão, os fluxos de uso da aplicação são descritos textualmente e seus resultados demonstrados visualmente através de imagens.}\\


\section{Interfaces gráficas}
\chapter{Considerações Finais}
\label{cap:05}

Texto das conclusões.

\textbf{Obs:} Este capítulo deve ser intitulado Considerações Finais em trabalhos de graduação para a Validação de Projeto de TCC. 

\section{Conclusões}

\section{Proposta de Trabalhos Futuros}
%\chapter{Cronograma}
\label{cap:06}

\textbf{Obs:} Este capítulo deve ser elaborado e estar contido em trabalhos de graduação para a Validação de Projeto de TCC. Na Avaliação Final de TCC (DEFESA) este capítulo não deve existir, visto que não haverá atividades após a Avaliação Final.

Segue abaixo o cronograma das atividades que serão executadas até a Avaliação Final de TCC.

\textbf{Obs:} Para facilitar, crie o cronograma usando o modelo do Word contido no projeto (imagens/templateCronograma.docx), ou qualquer outro \textit{software}, salve a imagem e atualize o arquivo imagens/cronograma.png.

\FloatBarrier
\begin{figure*}[!htbp]
	\centering
	\includegraphics[scale=1]{imagens/cronograma}
\end{figure*}
\FloatBarrier

\begin{enumerate}
	\item Descrição da atividade 1;
	\item Descrição da atividade 2;
	\item Descrição da atividade 3;
	\item Descrição da atividade 4;
	\item Descrição da atividade 5.
\end{enumerate}




% --- 
% Elementos Pós Textuais 
% ---
\postextual

% ---
% Referências bibliográficas
% ---
\bibliography{referencias}

% ---
% Glossário
% ---
%
% Consulte o manual da classe abntex2 para orientações sobre o glossário.
%\glossary

% ---
% Apêndices
% ---
% Texto ou documento elaborado pelo autor, a fim de complementar sua argumentação, sem prejuízo da unidade nuclear do trabalho.

% ---
% Inicia os apêndices
% ---
\begin{apendicesenv}
	
	% Imprime uma página indicando o início dos apêndices
	\partapendices* % o * serve para não mostrar a página %Apendices no Sumário
	
	% ----------------------------------------------------------
	\chapter{Título do Apêndice A}
	% ----------------------------------------------------------
	
	Texto do Apêndice A.
	
	
	
	% ----------------------------------------------------------
	\chapter{Título do Apêndice B}
	% ----------------------------------------------------------
	
	Texto do Apêndice B.
	

	
	
	
\end{apendicesenv}
% ---

% ---
% Anexos
% ---
% Texto ou documento não elaborado pelo autor, que serve de fundamentação, comprovação e ilustração.

% ---
% Inicia os anexos
% ---
\begin{anexosenv}
	
	% Imprime uma página indicando o início dos anexos
	\partanexos* % o * serve para não mostrar a página Anexos no %Sumário

	% ----------------------------------------------------------
	\chapter{Título do Anexo A}
	% ----------------------------------------------------------
	
	Texto do Anexo A.
	
	
	
	% ----------------------------------------------------------
	\chapter{Título do Anexo B}
	% ----------------------------------------------------------
	
	Texto do Anexo B.
	
	
	
	% ----------------------------------------------------------
	\chapter{Título do Anexo C}
	% ----------------------------------------------------------
	
	Texto do Anexo C.
	
	
	
\end{anexosenv}


% ---
% Inserir um percentual % na frente para que não apareça na versão final
\chapter{Exemplos}
\label{cap:99}

Texto considerando a revisão da literatura pertinente, dividido em seções e subseções.

Este é um exemplo de como usar figuras. Referência cruzada: Figura~\ref{fig:exemplo}

\FloatBarrier
\begin{figure}[!htbp]
	\centering
	\caption{Exemplo de figura}
	%scale redimensiona a figura.
	%1.5 = 150% do tamanho original
	%1 = 100% do tamanho original
	%0.20 = 20% do tamanho original
	\fbox{\includegraphics[scale=1]{imagens/IFSP-BRA.png}}
	\legend{Fonte: Disponível em: http://bra.ifsp.edu.br. Acesso em: 27 mar. 2020}
        %\legend{Fonte: Autor (ano) ou \citetext{chave}}
	\label{fig:exemplo}
\end{figure}
\FloatBarrier


Este é um exemplo de como usar tabelas. Referência cruzada: Tabela~\ref{tab:exemplo}

\FloatBarrier
\begin{table}[!htbp]
\centering
\caption{Exemplo de tabela de 2 colunas}
	\begin{tabular}{ c | c }
		\hline
		\textbf{Coluna 1} & \textbf{Coluna 2} \\ \hline
		Dado 1a           & Dado 2a           \\ \hline
		Dado 1b           & Dado 2b           \\ \hline
		Dado 1c           & Dado 2c           \\ \hline
		Dado 1d           & Dado 2d           \\ \hline
	\end{tabular}
	\\ \vspace{0.2cm}
	Fonte: Autoria própria (ano)
	\label{tab:exemplo}
\end{table}
\FloatBarrier


Este é um exemplo de como usar quadros. Referência cruzada: Quadro~\ref{qua:exemplo}

\FloatBarrier
\begin{quadro}[!htbp]
	\centering
	\caption{Exemplo de quadro}
	\includegraphics[scale=.7]{imagens/exemploQuadro}
	\\Fonte: Autoria própria (ano)
	\label{qua:exemplo}
\end{quadro}
\FloatBarrier

Este é um exemplo de como usar quadros. Referência cruzada: Quadro~\ref{tab:exemploquad}

\FloatBarrier
\begin{quadro}[!htbp]
\centering
\caption{Exemplo de Quadro de 3 colunas}
	\begin{tabular}{ | m{10em} | m{4cm}| m{4cm} | }
		\hline
		\textbf{Coluna 1} & \textbf{Coluna 2} & \textbf{Coluna 3} \\ \hline
		Dado 1a           & Dado 2a & \\ \hline
		Dado 1b           & Dado 2b & \\ \hline
		Dado 1c           & Dado 2c & \\ \hline
		Dado 1d           & Dado 2d & \\ \hline
	\end{tabular}
	\\ \vspace{0.2cm}
	Fonte: Autoria própria (ano)
	\label{tab:exemploquad}
\end{quadro}
\FloatBarrier

Este é um exemplo de como usar quadros. Referência cruzada: Quadro~\ref{tab:exemplo2}

\FloatBarrier
\begin{quadro}[!htbp]
\centering
\caption{Exemplo de Quadro de 2 colunas}
	\begin{tabular}{ | m{10em} | m{4cm}| }
		\hline
		\textbf{Coluna 1} & \textbf{Coluna 2}  \\ \hline
		Dado 1a           & Dado 2a  \\ \hline
		Dado 1b           & Dado 2b  \\ \hline
		Dado 1c           & Dado 2c  \\ \hline
		Dado 1d           & Dado 2d  \\ \hline
	\end{tabular}
	\\ \vspace{0.2cm}
	Fonte: Autoria própria (ano)
	\label{tab:exemplo2}
\end{quadro}
\FloatBarrier

Este é um exemplo de como usar equações. Referência cruzada: Equação~\ref{eq:exemplo}

\begin{equation}
\sum_{i=1}^{n} i = \frac{n(n+1)}{2}
\label{eq:exemplo}
\end{equation}

\clearpage

Exemplo de inserção de lista de código fonte (\textbf{\textcolor{red}{não use acentos no código!}}):

\lstinputlisting[language=Java]{fontes/ClasseExemplo.java} 



Este é um exemplo de como inserir texto sem formatação (ambiente verbatim):

\begin{verbatim}
	Texto sem formatação, como espaçamento igual.
\end{verbatim}


Exemplo de lista de itens:

\begin{itemize}
	\item \textbf{Item 1:} texto...;
	\item \textbf{Item 2:} texto...;
    \begin{itemize}
            \item \textbf{Subitem:} texto...;
            \item \textbf{Subitem:} texto...;
            \item \textbf{Subitem:} texto...;
        \end{itemize}
	\item \textbf{Item 3:} texto...;
	\item \textbf{Item n:} texto....
\end{itemize}


Exemplo de lista numerada:

\begin{enumerate}
	\item \textbf{Item:} texto...;
	\item \textbf{Item:} texto...;
    \begin{enumerate}
        \item \textbf{Subitem:} texto...;
        \item \textbf{Subitem:} texto...;
        \item \textbf{Subitem:} texto...;
    \end{enumerate}
	\item \textbf{Item:} texto...;
	\item \textbf{Item:} texto....
\end{enumerate}

Tipos de referência a serem utilizados no arquivo referencias.bib:
Exemplos de estruturas.\\

@article{<citation key>,
    author        = {},
    title         = {},
    journaltitle  = {},
    year          = {}
}

@online{<citation key>,
    author        = {},
    title         = {},
    year          = {},
    url           = {},
    urlaccessdate = {}
}

@book{<citation key>,
    author        = {},
    title         = {},
    year          = {}
}

@misc{<citation key>,
    author        = {},
    title         = {},
    year          = {},
    url           = {},
    urlaccessdate = {}
}


Exemplos de comandos para texto e referências:

\begin{itemize}
	\item Para iniciar um novo parágrafo, basta deixar uma linha em branco no código fonte;
	\item Não force o compilador a pular mais de uma linha, pois terá influência negativa na composição do documento;
	\item Sempre deixe o \LaTeX\ realizar a formatação de parágrafos e posicionamento de elementos;
	\item Utilização de aspas simples (abertura \verb|`|, fechamento \verb|'|): `Texto entre aspas simples';
	\item Utilização de aspas duplas (abertura \verb|``|, fechamento \verb|''|): ``Texto entre aspas duplas'';
	\item Negrito (comando \verb|\textbf|): \textbf{texto em negrito};
	\item Itálico (comando \verb|\textit|): \textit{texto em itálico};
	\item Sublinhado (comando \verb|\underline|): \underline{texto sublinhado};
	\item Negrito e itálico (usar comandos juntos): \textbf{\textit{texto em negrito e itálico}};
	\item Alterar cor do texto (comando \verb|\textcolor{cor}{texto}|):
	\begin{itemize}
		\item Exemplo \verb|\textcolor{red}{texto}|: \textcolor{red}{texto vermelho};
		\item Exemplo \verb|\textcolor[RGB]{255, 102, 0}|: \textcolor[RGB]{255, 102, 0}{texto laranja};
		\item Exemplo \verb|\textcolor[HTML]{006AD7}|: \textcolor[HTML]{006AD7}{texto azul};
	\end{itemize}
	\item Ambiente matemático inline (comando \verb|$ expressão $|): $s = x^2-2x +1$;
	\item Referência normal (comando \verb|\cite|):
	\begin{itemize}
		\item \cite{Agaisse1995};
		\item \cite{Abedi2014};
		\item \cite{Baum2016};
        
	\end{itemize}
	\item Referência normal com mais de uma obra (comando \verb|\cite|):
	\begin{itemize}
		\item \cite{Abedi2014, Agaisse1995};
		\item \cite{AgapitoTenfen2014, Baum2016, Nelson2014};
	\end{itemize}
	\item Referência nome e ano (comando \verb|\citeauthorandyear|):
	\begin{itemize}
		\item \citeauthorandyear{bervian2007a};
		\item \citeonline{bervian2007a};
		\item \citeauthorandyear{documento2018};
		\item \citeauthorandyear{Abedi2014};
        \item \citeonline{schaefer2022};
        \item \citeonline{castro2019};
	\end{itemize}
	\item Referência apud ({mais antigo} {mais novo}):\\
	\verb|\apud[pagina]{autor-indireto}{autor-direto}|):
	\begin{itemize}
		\item \apud{Agaisse1995}{Abedi2014};
		\item \apudonline{Agaisse1995}{Abedi2014};
	\end{itemize}
	\item Referência apud [pagina]({mais antigo} {mais novo}):\\
	\verb|\apud[pagina]{autor-indireto}{autor-direto}|):
	\begin{itemize}
		\item \apud[p. 81]{Agaisse1995}{Abedi2014};
		\item \apudonline[p. 81]{Agaisse1995}{Abedi2014};
	\end{itemize}
	
	\item Referência do mesmo autor, mesmo ano com obras distintas 
	\verb|\citeauthorandyear|):
	\begin{itemize}
		\item \citeauthorandyear{Agaisse1996a};
		\item \citeauthorandyear{Agaisse1996b};
		\item \cite{Agaisse1996a};
		\item \cite{Agaisse1996b};
	\end{itemize}
	
	
\end{itemize}


Exemplo 1 de citação direta:

\begin{citacao}
	Os 20 aminoácidos usualmente encontrados como resíduos em proteínas contém um grupo $\alpha$-carboxil, um grupo $\alpha$-amino e um grupo R distinto substituído no átomo de carbono $\alpha$. O átomo de carbono $\alpha$ de todos os aminoácidos, com exceção da glicina, é assimétrico e, portanto, os aminoácidos podem existir em pelo menos duas formas estereoisoméricas. Somente os estereoisômeros L, com uma configuração relacionada à configuração absoluta da molécula de referência L-gliceraldeído, são encontrados em proteínas \cite[p. 81]{Nelson2014}.
\end{citacao}

Exemplo 2 de citação direta:

\begin{citacao}
	\textit{These various insecticidal proteins are synthesized during the stationary phase and accumulate in the mother cell as a crystal inclusion which can account for up to 25\% of the dry weight of the sporulated cells. The amount of crystal protein produced by a B. thuringiensis culture in laboratory conditions (about 0.5 mg of protein per ml) and the size of the crystals (24) indicate that each cell has to synthesize $10^6$ to $2 \times 10^6$ $\delta$-endotoxin molecules during the stationary phase to form a crystal} \cite[p. 1]{Agaisse1995}.
\end{citacao}

Exemplo de nota de rodapé\footnote{Essa é uma nota de rodapé!}.

Exemplo de rodapé com referência indicada nele\footnote{\citetext{marciano2020}}

% ---

%---
% ÍNDICE REMISSIVO
%---
%\phantompart
%\printindex

% --- 
% Fim do Documento
% ---
\end{document}
